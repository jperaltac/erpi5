% !TEX program = lualatex
\documentclass[aspectratio=169,professionalfonts]{beamer}
%------------------------- Preamble (beamer) -------------------------%
%   Require LuaLaTeX + --shell-escape (for minted)                    %
%--------------------------------------------------------------------%
\usepackage[spanish]{babel}

% ---------- Fonts (modern look) ----------
\usepackage{fontspec}
\defaultfontfeatures{Ligatures=TeX, Scale=MatchLowercase}
\setmainfont{Inter}       % Sans‑serif body (install ttf–Inter)
\setsansfont{Inter}
\setmonofont{Fira Code}   % Nice mono for code

% ---------- Theme ----------
\usetheme[
  titleformat=smallcaps,
  numbering=fraction,
  progressbar=frametitle   % thin progress line under the title
]{metropolis}              % modern beamer theme
\metroset{block=fill}      % coloured blocks

% ---------- Colours ----------
\definecolor{clblue}{HTML}{005F9E}
\definecolor{clorange}{HTML}{FF9130}
\definecolor{clgrey}{HTML}{F4F4F4}

% ---------- minted ----------
\usepackage{minted}
\setminted{
  fontsize=\footnotesize,
  autogobble,
  breaklines,
  tabsize=2,
  style=monokai,
  linenos
}

% ---------- Custom boxes (tcolorbox) ----------
\usepackage{tcolorbox}
\tcbuselibrary{skins, breakable}
% Warning box
\newtcolorbox{warnbox}{
  colback=clorange!10,
  colframe=clorange!80!black,
  breakable,
  title={\faWarning\quad Advertencia},
  fonttitle=\bfseries,
  left=2mm, right=2mm, top=1mm, bottom=1mm
}
% Information box
\newtcolorbox{infobox}{
  colback=clblue!5,
  colframe=clblue!60!black,
  breakable,
  title={\faInfoCircle\quad Nota},
  fonttitle=\bfseries,
  left=2mm, right=2mm, top=1mm, bottom=1mm
}

% ---------- Icons ----------
\usepackage{fontawesome5}

% ---------- Misc tweaks ----------
\graphicspath{{figures/}}
\setlength{\parskip}{0.5em}
\addto\extrasspanish{\renewcommand{\contentsname}{Índice}}
%--------------------------------------------------------------------%



\title[ClústerLab • Día 4]{Montaje físico y escaneo de red}
\subtitle{Conectando todas las Raspberry Pi al switch}
\author{Equipo docente ClústerLab}
\date{11 de agosto de 2025}

\begin{document}

%------------------------------ portada ----------------------------
\begin{frame}[plain]
  \titlepage
\end{frame}

%------------------------------ objetivos --------------------------
\begin{frame}[fragile]{Objetivos de la sesión}
  \begin{itemize}
    \item Montar físicamente el clúster de Raspberry Pi y etiquetar conexiones.
    \item Verificar el enlace Gigabit y actividad con los LED del switch.
    \item Escanear la red para identificar IP y hostname de cada nodo.
    \item Registrar la tabla de nodos en un archivo compartido.
  \end{itemize}
\end{frame}

%------------------------------ agenda -----------------------------
\begin{frame}[fragile]{Agenda (09:00--11:00)}
  \begin{enumerate}
    \item Distribución de hardware y numeración de nodos.
    \item Conectar cables Ethernet y encender las Pi.
    \item Escaneo con \texttt{arp-scan} y \texttt{nmap}.
    \item Generar lista \texttt{hosts.csv} para el clúster.
    \item Actividad práctica grupal.
  \end{enumerate}
\end{frame}

%------------------------------ diagrama ---------------------------
\begin{frame}[fragile]{Topología física sugerida}
  \begin{center}
    \includegraphics[width=0.8\linewidth]{example-image} % reemplazar por diagrama real
  \end{center}
  \begin{itemize}
    \item Un switch Gigabit de 24 puertos.
    \item Nodo \textbf{head} conectado al puerto 1.
    \item Nodos de cálculo en puertos 2 -- 18.
  \end{itemize}
\end{frame}

%------------------------------ leds -------------------------------
\begin{frame}[fragile]{Comprobación de enlace}
  \begin{itemize}
    \item LED verde fijo \(\Rightarrow\) enlace 1 Gb/s.
    \item LED parpadeante \(\Rightarrow\) tráfico de datos.
  \end{itemize}
  \pause
  \textbf{Comando de confirmación}
  \begin{minted}{bash}
$ ethtool eth0 | grep Speed
  \end{minted}
\end{frame}

%------------------------------ arp-scan ---------------------------
\begin{frame}[fragile]{Descubrir dispositivos con arp-scan}
  \begin{minted}{bash}
$ sudo apt install arp-scan
$ sudo arp-scan --interface=eth0 10.0.0.0/24 | grep Raspberry
  \end{minted}
  \vspace{0.5em}
  \begin{itemize}
    \item Identifica MAC y IP rápidamente.
    \item Exporta a CSV:
      \begin{minted}{bash}
$ arp-scan --quiet --format=csv 10.0.0.0/24 > scan.csv
      \end{minted}
  \end{itemize}
\end{frame}

%------------------------------ nmap -------------------------------
\begin{frame}[fragile]{Verificar puertos abiertos con nmap}
  \begin{minted}{bash}
$ sudo nmap -p 22 --open 10.0.0.0/24
  \end{minted}
  \begin{itemize}
    \item Confirma que el servicio SSH está activo.
  \end{itemize}
\end{frame}

%------------------------------ hosts.csv --------------------------
\begin{frame}[fragile]{Ejemplo de archivo \texttt{hosts.csv}}
  \begin{minted}{text}
IP,Hostname,Grupo
10.0.0.21,pi21,Grupo1
10.0.0.22,pi22,Grupo1
...
  \end{minted}
  Comparte este archivo en el repositorio NFS (próxima sesión).
\end{frame}

%------------------------------ actividad --------------------------
\begin{frame}[fragile]{Actividad práctica (30 min)}
  \begin{enumerate}
    \item Conecta tus dos Pi y verifica el enlace.
    \item Usa \texttt{arp-scan} para obtener IP y hostname.
    \item Rellena \texttt{hosts.csv} y súbelo al canal del curso.
    \item Prueba \texttt{ssh piXX} entre nodos.
  \end{enumerate}
\end{frame}

%------------------------------ resumen ----------------------------
\begin{frame}[fragile]{Resumen de la sesión}
  \begin{itemize}
    \item Clúster montado y cableado correctamente.
    \item Tabla de nodos lista para automatizar configuraciones.
  \end{itemize}
\end{frame}

%------------------------------ recursos ---------------------------
\begin{frame}[fragile]{Recursos recomendados}
  \begin{itemize}
    \item \url{https://github.com/royhills/arp-scan}.
    \item \url{https://nmap.org/book/inst-windows.html}.
  \end{itemize}
\end{frame}

\end{document}
