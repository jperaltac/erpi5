% !TEX program = lualatex
\documentclass[aspectratio=169,professionalfonts]{beamer}
%------------------------- Preamble (beamer) -------------------------%
%   Require LuaLaTeX + --shell-escape (for minted)                    %
%--------------------------------------------------------------------%
\usepackage[spanish]{babel}

% ---------- Fonts (modern look) ----------
\usepackage{fontspec}
\defaultfontfeatures{Ligatures=TeX, Scale=MatchLowercase}
\setmainfont{Inter}       % Sans‑serif body (install ttf–Inter)
\setsansfont{Inter}
\setmonofont{Fira Code}   % Nice mono for code

% ---------- Theme ----------
\usetheme[
  titleformat=smallcaps,
  numbering=fraction,
  progressbar=frametitle   % thin progress line under the title
]{metropolis}              % modern beamer theme
\metroset{block=fill}      % coloured blocks

% ---------- Colours ----------
\definecolor{clblue}{HTML}{005F9E}
\definecolor{clorange}{HTML}{FF9130}
\definecolor{clgrey}{HTML}{F4F4F4}

% ---------- minted ----------
\usepackage{minted}
\setminted{
  fontsize=\footnotesize,
  autogobble,
  breaklines,
  tabsize=2,
  style=monokai,
  linenos
}

% ---------- Custom boxes (tcolorbox) ----------
\usepackage{tcolorbox}
\tcbuselibrary{skins, breakable}
% Warning box
\newtcolorbox{warnbox}{
  colback=clorange!10,
  colframe=clorange!80!black,
  breakable,
  title={\faWarning\quad Advertencia},
  fonttitle=\bfseries,
  left=2mm, right=2mm, top=1mm, bottom=1mm
}
% Information box
\newtcolorbox{infobox}{
  colback=clblue!5,
  colframe=clblue!60!black,
  breakable,
  title={\faInfoCircle\quad Nota},
  fonttitle=\bfseries,
  left=2mm, right=2mm, top=1mm, bottom=1mm
}

% ---------- Icons ----------
\usepackage{fontawesome5}

% ---------- Misc tweaks ----------
\graphicspath{{figures/}}
\setlength{\parskip}{0.5em}
\addto\extrasspanish{\renewcommand{\contentsname}{Índice}}
%--------------------------------------------------------------------%



\title[ClústerLab • Día 4]{Servidor NFS y montaje en nodos}
\subtitle{Compartiendo datos en el clúster}
\author{Equipo docente ClústerLab}
\date{11 de agosto de 2025}

\begin{document}

%------------------------------ portada ----------------------------
\begin{frame}[plain]
  \titlepage
\end{frame}

%------------------------------ objetivos --------------------------
\begin{frame}[fragile]{Objetivos de la sesión}
  \begin{itemize}
    \item Instalar y habilitar \texttt{nfs-kernel-server} en el nodo head.
    \item Exportar un directorio compartido mediante \texttt{/etc/exports}.
    \item Montar el recurso en los nodos con \texttt{nfs-common}.
    \item Validar lectura y escritura con un script Python simple.
  \end{itemize}
\end{frame}

%------------------------------ agenda -----------------------------
\begin{frame}[fragile]{Agenda (11:30 – 13:00)}
  \begin{enumerate}
    \item Instalación del servidor NFS.
    \item Configuración de \texttt{/srv/nfs/data} y opciones de exportación.
    \item Montaje temporal con \texttt{mount}.
    \item Persistencia en \texttt{/etc/fstab}.
    \item Prueba de rendimiento y permisos.
    \item Actividad práctica y cierre.
  \end{enumerate}
\end{frame}

%------------------------------ install nfs ------------------------
\begin{frame}[fragile]{Instalar el servidor NFS (nodo head)}
  \begin{minted}{bash}
$ sudo apt install nfs-kernel-server
$ sudo mkdir -p /srv/nfs/data
$ sudo chown pi:pi /srv/nfs/data
  \end{minted}
  \begin{block}{Abrir el servicio}
    \begin{minted}{bash}
$ sudo systemctl enable --now nfs-server
    \end{minted}
  \end{block}
\end{frame}

%------------------------------ exports ---------------------------
\begin{frame}[fragile]{Configurar \texttt{/etc/exports}}
  \begin{minted}{bash}
/srv/nfs/data 10.0.0.0/24(rw,sync,no_root_squash)
  \end{minted}
  \begin{minted}{bash}
$ sudo exportfs -rav
  \end{minted}
  \begin{itemize}
    \item \texttt{rw}: lectura y escritura.
    \item \texttt{sync}: escribe datos de inmediato (más seguro).
  \end{itemize}
\end{frame}

%------------------------------ client install --------------------
\begin{frame}[fragile]{Instalar cliente NFS (nodos)}
  \begin{minted}{bash}
$ sudo apt install nfs-common
$ sudo mkdir -p /mnt/shared
  \end{minted}
\end{frame}

%------------------------------ mount -----------------------------
\begin{frame}[fragile]{Montaje temporal}
  \begin{minted}{bash}
$ sudo mount -t nfs head:/srv/nfs/data /mnt/shared
  \end{minted}
  \begin{itemize}
    \item Comprueba con \texttt{df -h | grep nfs}.
  \end{itemize}
\end{frame}

%------------------------------ fstab -----------------------------
\begin{frame}[fragile]{Montaje permanente en \texttt{/etc/fstab}}
  \begin{minted}{text}
head:/srv/nfs/data  /mnt/shared  nfs  defaults,_netdev  0  0
  \end{minted}
  \begin{minted}{bash}
$ sudo mount -a  # prueba sintaxis
  \end{minted}
\end{frame}

%------------------------------ python test -----------------------
\begin{frame}[fragile]{Script Python de prueba}
  \begin{minted}[bgcolor=gray!15]{python}
from pathlib import Path
import socket
root = Path('/mnt/shared')
file = root / f"hello_{socket.gethostname()}.txt"
file.write_text("¡Hola desde Raspberry Pi!")
print("Escrito:", file)
print("Contenido:", file.read_text())
  \end{minted}
  Ejecutar en al menos dos nodos y verificar que ambos archivos aparezcan.
\end{frame}

%------------------------------ actividad -------------------------
\begin{frame}[fragile]{Actividad práctica (30 min)}
  \begin{enumerate}
    \item Exporta \texttt{/srv/nfs/data} desde el nodo head.
    \item Monta en \texttt{/mnt/shared} en tus nodos.
    \item Corre el script Python y confirma acceso simultáneo.
    \item Subir una captura de pantalla de \texttt{ls /mnt/shared} al grupo.
  \end{enumerate}
\end{frame}

%------------------------------ resumen ---------------------------
\begin{frame}[fragile]{Resumen de la sesión}
  \begin{itemize}
    \item NFS habilita almacenamiento centralizado fácil de configurar.
    \item Montaje persistente via \texttt{/etc/fstab} evita comandos manuales.
    \item Pruebas de lectura/escritura aseguran permisos correctos.
  \end{itemize}
\end{frame}

%------------------------------ recursos --------------------------
\begin{frame}[fragile]{Recursos recomendados}
  \begin{itemize}
    \item \url{https://wiki.archlinux.org/title/NFS} (guía completa).
    \item \url{https://help.ubuntu.com/lts/serverguide/serverguide.pdf} sección NFS.
  \end{itemize}
\end{frame}

\end{document}
