% !TEX program = lualatex
\documentclass[aspectratio=169,professionalfonts]{beamer}
%------------------------- Preamble (beamer) -------------------------%
%   Require LuaLaTeX + --shell-escape (for minted)                    %
%--------------------------------------------------------------------%
\usepackage[spanish]{babel}

% ---------- Fonts (modern look) ----------
\usepackage{fontspec}
\defaultfontfeatures{Ligatures=TeX, Scale=MatchLowercase}
\setmainfont{Inter}       % Sans‑serif body (install ttf–Inter)
\setsansfont{Inter}
\setmonofont{Fira Code}   % Nice mono for code

% ---------- Theme ----------
\usetheme[
  titleformat=smallcaps,
  numbering=fraction,
  progressbar=frametitle   % thin progress line under the title
]{metropolis}              % modern beamer theme
\metroset{block=fill}      % coloured blocks

% ---------- Colours ----------
\definecolor{clblue}{HTML}{005F9E}
\definecolor{clorange}{HTML}{FF9130}
\definecolor{clgrey}{HTML}{F4F4F4}

% ---------- minted ----------
\usepackage{minted}
\setminted{
  fontsize=\footnotesize,
  autogobble,
  breaklines,
  tabsize=2,
  style=monokai,
  linenos
}

% ---------- Custom boxes (tcolorbox) ----------
\usepackage{tcolorbox}
\tcbuselibrary{skins, breakable}
% Warning box
\newtcolorbox{warnbox}{
  colback=clorange!10,
  colframe=clorange!80!black,
  breakable,
  title={\faWarning\quad Advertencia},
  fonttitle=\bfseries,
  left=2mm, right=2mm, top=1mm, bottom=1mm
}
% Information box
\newtcolorbox{infobox}{
  colback=clblue!5,
  colframe=clblue!60!black,
  breakable,
  title={\faInfoCircle\quad Nota},
  fonttitle=\bfseries,
  left=2mm, right=2mm, top=1mm, bottom=1mm
}

% ---------- Icons ----------
\usepackage{fontawesome5}

% ---------- Misc tweaks ----------
\graphicspath{{figures/}}
\setlength{\parskip}{0.5em}
\addto\extrasspanish{\renewcommand{\contentsname}{Índice}}
%--------------------------------------------------------------------%



\title[ClústerLab • Día 1]{Linux básico y primeros comandos}
\subtitle{Hands--on en Raspberry Pi 5}
\author{Equipo docente ClústerLab}
\date{4 de agosto de 2025}

\begin{document}

%---------------- title ----------------------------%
\begin{frame}[plain]
  \titlepage
\end{frame}

\begin{frame}[fragile]{Primer arranque en aula}
\begin{itemize}
  \item Usaremos \textbf{una} Raspberry Pi por equipo, con \textbf{pantalla, teclado y mouse}.
  \item Objetivos del día:
  \begin{itemize}
    \item Navegar el sistema de archivos y la terminal.
    \item Instalar utilidades básicas.
    \item Dejar la Pi \textit{lista} para el día 2 (SSH y trabajo remoto).
  \end{itemize}
\end{itemize}
\begin{minted}{bash}
# Usuario por defecto (si aplica a tu imagen)
user: pi    pass: raspberry
\end{minted}
\begin{infobox}
No activaremos SSH hoy; lo veremos al final y se usará desde el día 2.
\end{infobox}
\end{frame}

%---------------- estructura FS ------------------------------%
\begin{frame}[fragile]{Estructura del sistema de archivos}
\begin{columns}[T,onlytextwidth]
\column{0.55\linewidth}
\begin{minted}{bash}
/                    # raíz
/home/pi             # tu carpeta
/etc                 # configuración del sistema
/var/log             # logs
/opt                 # software opcional
/tmp                 # temporales
/boot                # arranque (Raspberry)
\end{minted}
\begin{minted}{bash}
$ ls -l /
$ tree -L 1 /home/pi   # requiere 'tree'
\end{minted}
\column{0.45\linewidth}
\begin{infobox}
Carpetas “del sistema” (\texttt{/etc}, \texttt{/var}) vs “tu espacio” (\texttt{/home/pi}).
\end{infobox}
\end{columns}
\end{frame}

%---------------- comandos esenciales 2 ----------------------%
\begin{frame}[fragile]{Comandos esenciales (ampliado)}
\begin{minted}{bash}
$ ls -lah         # tamaños humanos, muestra ocultos
$ cd ..           # subir un nivel
$ mv a b          # renombrar o mover
$ cp a b          # copiar
$ cp -r dir1 dir2 # copiar recursivo
$ history         # historial de comandos
$ which python3   # ruta de un ejecutable
$ clear           # limpiar pantalla
\end{minted}
\begin{infobox}
Atajos: \texttt{Ctrl+C} (cancelar), \texttt{Ctrl+A/E} (inicio/fin),
\texttt{Ctrl+R} (buscar historial), \texttt{Tab} (autocompletar).
\end{infobox}
\end{frame}

%---------------- redirecciones & pipes avanzado -------------%
\begin{frame}[fragile]{Redirecciones y pipes (útilísimo)}
\begin{minted}{bash}
$ comando > out.txt            # stdout a archivo (sobrescribe)
$ comando >> out.txt           # agrega al final
$ comando 2> err.txt           # stderr a archivo
$ comando 2>&1 | tee log.txt   # mezcla y copia a pantalla+archivo
$ dmesg | grep -i usb | less   # filtrar y paginar
$ du -sh * | sort -h           # uso de espacio ordenado
\end{minted}
\end{frame}

%---------------- globbing, find, grep -----------------------%
\begin{frame}[fragile]{Búsquedas: globbing, \texttt{find} y \texttt{grep}}
\begin{minted}{bash}
$ ls *.txt                       # comodines
$ find . -name "*.py"            # buscar por nombre
$ find /var/log -type f -mtime -1   # logs de hoy
$ grep -R "ERROR" /var/log/      # buscar texto recursivo
$ grep -n "main(" *.c            # con número de línea
\end{minted}
\begin{warnbox}
Cuidado con \texttt{find / -name ...} sin permisos: puede tardar mucho.
\end{warnbox}
\end{frame}

%---------------- permisos (ampliado) ------------------------%
\begin{frame}[fragile]{Permisos (ampliado)}
\begin{minted}{bash}
$ ls -l backup.sh
-rwxr-xr-- 1 pi pi  54 jul  8 12:30 backup.sh

$ chmod u+x script.sh     # ejecutar para dueño
$ chmod g+w datos/        # escritura para grupo
$ chgrp alumnos datos/    # cambiar grupo
$ umask                   # máscara por defecto
\end{minted}
\begin{warnbox}
Evita \texttt{chmod 777}. Permisos mínimos necesarios.
\end{warnbox}
\end{frame}

%---------------- usuarios y sudo ----------------------------%
\begin{frame}[fragile]{Usuarios, grupos y \texttt{sudo}}
\begin{minted}{bash}
$ whoami
$ id                      # UID/GID y grupos
$ groups pi
$ sudo -v                 # validar sudo
$ sudo adduser alumno
$ sudo usermod -aG sudo alumno
\end{minted}
\begin{infobox}
Trabaja con usuario normal; usa \texttt{sudo} sólo cuando haga falta.
\end{infobox}
\end{frame}

%---------------- variables y entorno ------------------------%
\begin{frame}[fragile]{Variables de entorno y alias}
\begin{minted}{bash}
$ echo $HOME  $USER  $SHELL
$ echo $PATH
$ export EDITOR=nano

# alias útiles (añádelos a ~/.bashrc)
alias ll='ls -lah'
alias gs='git status'

$ source ~/.bashrc   # recargar configuración
\end{minted}
\end{frame}

%---------------- editores ----------------------------%
\begin{frame}[fragile]{Editores rápidos: \texttt{nano}, \texttt{micro}, \texttt{vim}}
\begin{minted}{bash}
$ sudo apt update
$ sudo apt install -y nano micro vim
$ nano notas.txt     # controles en pantalla
$ micro notas.txt    # similar a editor moderno
$ vim notas.txt      # potente (requiere práctica)
\end{minted}
\begin{infobox}
Para empezar, \texttt{nano} o \texttt{micro} son ideales.
\end{infobox}
\end{frame}

%---------------- gestión de paquetes ------------------------%
\begin{frame}[fragile]{Gestión de paquetes con \texttt{apt}}
\begin{minted}{bash}
$ sudo apt update
$ sudo apt upgrade -y
$ sudo apt install -y htop tree git curl wget zip unzip
$ sudo apt install -y build-essential cmake pkg-config
$ sudo apt install -y python3-pip python3-venv
$ sudo apt autoremove -y
\end{minted}
\begin{infobox}
\texttt{build-essential} instala compiladores y herramientas clave.
\end{infobox}
\end{frame}

%---------------- monitoreo del sistema ----------------------%
\begin{frame}[fragile]{Monitoreo de sistema}
\begin{minted}{bash}
$ htop                     # procesos interactivo
$ free -h                  # memoria
$ df -h                    # uso de disco
$ lsblk                    # dispositivos de bloque
$ vcgencmd measure_temp    # temperatura (Raspberry)
$ watch -n 1 "vcgencmd measure_temp"
\end{minted}
\end{frame}

%---------------- redes básico -------------------------------%
\begin{frame}[fragile]{Red básica}
\begin{minted}{bash}
$ hostname            # nombre de equipo
$ hostname -I         # IPs
$ ip a                # interfaces
$ ping -c 4 8.8.8.8   # conectividad
$ ping -c 4 google.com
$ sudo apt install -y net-tools
$ ifconfig            # (alternativa, en net-tools)
\end{minted}
\begin{infobox}
Nombrar tu equipo ayuda: \texttt{sudo hostnamectl set-hostname pi-01}
\end{infobox}
\end{frame}

%---------------- transferencia de archivos ------------------%
\begin{frame}[fragile]{Transferencia de archivos (día 2 con SSH)}
\begin{minted}{bash}
# scp: copia simple
$ scp archivo.txt pi@pi-01:~/    # a la Pi
$ scp pi@pi-01:~/datos.csv .     # desde la Pi

# rsync: sincronización eficiente
$ rsync -avh carpeta/ pi@pi-01:~/carpeta/
\end{minted}
\begin{infobox}
Usaremos esto cuando activemos SSH mañana.
\end{infobox}
\end{frame}

%---------------- tmux/screen --------------------------------%
\begin{frame}[fragile]{Sesiones persistentes: \texttt{tmux}}
\begin{minted}{bash}
$ sudo apt install -y tmux
$ tmux new -s trabajo     # crear sesión
# dentro: Ctrl+B, luego D  -> detach
$ tmux ls                  # listar sesiones
$ tmux attach -t trabajo   # volver a entrar
\end{minted}
\begin{infobox}
Ideal para dejar procesos corriendo sin perderlos al cerrar la terminal.
\end{infobox}
\end{frame}

%---------------- Python + Jupyter ---------------------------%
\begin{frame}[fragile]{Python y Jupyter (opcional)}
\begin{minted}{bash}
$ python3 --version
$ python3 -m venv ~/venvs/base
$ source ~/venvs/base/bin/activate
(base) $ pip install --upgrade pip
(base) $ pip install jupyterlab numpy matplotlib
(base) $ jupyter lab --no-browser --ip=0.0.0.0
\end{minted}
\begin{infobox}
Con SSH haremos \textit{port forwarding} para abrir Jupyter desde tu laptop.
\end{infobox}
\end{frame}

%---------------- Git básico ---------------------------------%
\begin{frame}[fragile]{Git básico}
\begin{minted}{bash}
$ git config --global user.name "Tu Nombre"
$ git config --global user.email "tu@correo"
$ mkdir demo && cd demo
$ git init
$ echo "hola" > README.md
$ git add README.md
$ git commit -m "Primer commit"
\end{minted}
\end{frame}

%---------------- raspi-config -------------------------------%
\begin{frame}[fragile]{\texttt{raspi-config}: ajustes rápidos}
\begin{minted}{bash}
$ sudo raspi-config
# Cambiar localización, zona horaria, expandir filesystem,
# habilitar interfaces (I2C, SPI, cámara), etc.
\end{minted}
\begin{warnbox}
Aplica cambios con cuidado; anota lo que modificas.
\end{warnbox}
\end{frame}

%---------------- utilidades recomendadas --------------------%
\begin{frame}[fragile]{Utilidades recomendadas en Raspberry Pi}
\begin{minted}{bash}
$ sudo apt install -y neofetch ranger ncdu nmap \
  screenfetch mc jq ripgrep btop
$ neofetch       # resumen del sistema
$ ncdu           # uso de disco interactivo
$ ranger         # explorador en terminal
\end{minted}
\begin{infobox}
\texttt{ncdu} y \texttt{ranger} ayudan a manejar archivos y espacio.
\end{infobox}
\end{frame}

%---------------- seguridad básica ---------------------------%
\begin{frame}[fragile]{Seguridad básica (hoy, local)}
\begin{minted}{bash}
$ passwd                              # cambia tu contraseña
$ sudo apt update && sudo apt upgrade -y
$ sudo apt install -y unattended-upgrades
$ sudo dpkg-reconfigure unattended-upgrades
\end{minted}
\begin{infobox}
Mantén tu sistema actualizado y evita usar \texttt{sudo} sin necesidad.
\end{infobox}
\end{frame}

%---------------- SSH (día 2) --------------------------------%
\begin{frame}[fragile]{SSH: activar servicio (día 2)}
\begin{minted}{bash}
$ sudo systemctl enable --now ssh
$ systemctl status ssh
$ sudo ss -ltnp | grep ':22'   # puerto en escucha
\end{minted}
\begin{warnbox}
Activa SSH sólo cuando controles la red del aula.
\end{warnbox}
\end{frame}

%---------------- SSH (claves) -------------------------------%
\begin{frame}[fragile]{SSH: claves y acceso sin contraseña}
\begin{minted}{bash}
# En tu laptop
$ ssh-keygen -t ed25519 -C "tu@correo"
$ ssh-copy-id pi@pi-01        # o pi@IP.DE.TU.PI
$ ssh pi@pi-01
\end{minted}
\begin{minted}{bash}
# Config opcional (~/.ssh/config en tu laptop)
Host pi-01
  HostName 192.168.1.50
  User pi
  IdentityFile ~/.ssh/id_ed25519
\end{minted}
\end{frame}

%---------------- SSH (túneles y scp/rsync) ------------------%
\begin{frame}[fragile]{SSH: túneles y transferencia}
\begin{minted}{bash}
# Port forwarding (Jupyter en la Pi -> tu laptop)
$ ssh -L 8888:localhost:8888 pi@pi-01

# Copia de archivos
$ scp archivo.txt pi@pi-01:~/     # subir
$ scp pi@pi-01:~/datos.csv .      # bajar

# Sincronización eficiente
$ rsync -avh carpeta/ pi@pi-01:~/carpeta/
\end{minted}
\end{frame}

%---------------- diagnóstico SSH ----------------------------%
\begin{frame}[fragile]{SSH: diagnóstico rápido}
\begin{minted}{bash}
$ ping -c 2 pi-01
$ ssh -vvv pi@pi-01          # modo verboso
$ sudo journalctl -u ssh -e  # logs en la Pi
$ sudo ufw status            # firewall (si está activo)
\end{minted}
\begin{infobox}
Si falla por nombre, intenta por IP directa.
\end{infobox}
\end{frame}


%---------------- actividad ------------------------%
\begin{frame}{Actividad práctica (30 min)}
\begin{enumerate}
  \item Crear \texttt{~/proyectos/\{scripts,datos,logs\}}
  \item Escribir \texttt{backup.sh} que:
    \begin{itemize}
      \item reciba carpeta (\$1)
      \item comprima con \texttt{tar} \(\to\) \texttt{.tar.gz}
      \item mueva a \texttt{~/backups}
    \end{itemize}
  \item Añadir bit de ejecución y probar.
\end{enumerate}

\small\faLightbulb\; Tip: usa \texttt{\$(date +\%F)} para sellar fecha.
\end{frame}

%---------------- cheatsheet -----------------------%
\begin{frame}[fragile]{Mini cheatsheet}
\begin{columns}[T,onlytextwidth]
\column{0.50\linewidth}
\begin{itemize}
  \item \textbf{Procesos}: \mintinline{bash}{htop}, \mintinline{bash}{ps aux}
  \item \textbf{Disco}: \mintinline{bash}{df -h}, \mintinline{bash}{du -sh *}
  \item \textbf{Buscar}: \mintinline{bash}{grep}, \mintinline{bash}{find}
  \item \textbf{Red}: \mintinline{bash}{ip a}, \mintinline{bash}{ping}
\end{itemize}

\column{0.50\linewidth}
\begin{itemize}
  \item \textbf{Editar}: \mintinline{bash}{nano}, \mintinline{bash}{vim}
  \item \textbf{Actualizar}: \mintinline{bash}{sudo apt update && apt upgrade}
  \item \textbf{Historial}: \mintinline{bash}{history}, \mintinline{bash}{!42}
  \item \textbf{Autocompletar}: ↹ (TAB)
\end{itemize}
\end{columns}
\end{frame}

%---------------- recursos -------------------------%
\begin{frame}{Recursos recomendados}
\begin{itemize}
  \item \emph{The Linux Command Line}, W. Shotts Jr. (ebook gratuito)
  \item \url{https://linuxjourney.com} -- interactivo
  \item Documentación oficial de Raspberry Pi OS
\end{itemize}
\end{frame}

%---------------- cierre ---------------------------%
\begin{frame}[plain]
  \centering \Huge ¡Buen trabajo!\\[1ex]
  Siguiente bloque: redes IP y claves SSH
\end{frame}

\end{document}
