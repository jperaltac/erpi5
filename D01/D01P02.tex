% !TEX program = lualatex
\documentclass[aspectratio=169, professionalfonts]{beamer}
\usepackage[spanish]{babel}
\usepackage{fontspec}
\usepackage{graphicx}
\usepackage{xcolor}
\usepackage{listings}
\usepackage{emoji}
\setmainfont{Libertinus Serif}

%--------- listings (estilo shell) -----------------
\lstdefinestyle{shell}{
  language=bash,
  basicstyle=\ttfamily\small,
  keywordstyle=\color{blue!70!black},
  commentstyle=\color{gray},
  showstringspaces=false
}

\title[ClústerLab • Día 1]{Linux básico y primeros comandos}
\subtitle{Hands-on en Raspberry Pi 5}
\author{Equipo docente ClústerLab}
\date{4 de agosto 2025}

\begin{document}

%---------------- title ----------------------------
\begin{frame}[plain]
  \titlepage
\end{frame}

%---------------- SSH ------------------------------
\begin{frame}[fragile]{Conectar a la Raspberry Pi}
\begin{block}{Con pantalla y teclado}
  Usuario predeterminado: \texttt{pi} \hfill Contraseña: \texttt{raspberry}
\end{block}
\pause
\begin{block}{Desde tu laptop (SSH)}
\begin{lstlisting}[style=shell]
$ ssh pi@192.168.1.50      # IP de tu Pi
# Primer login → cambiar contraseña
\end{lstlisting}
\end{block}
\pause
\small Si usas Windows, realiza la conexión desde Ubuntu WSL 2 o usa PuTTY.
\end{frame}

%---------------- comandos esenciales --------------
\begin{frame}[fragile]{Comandos esenciales}
\begin{columns}[T]
\begin{column}{0.55\textwidth}
\begin{lstlisting}[style=shell]
$ pwd              # ruta actual
$ ls -lh           # listar (human-readable)
$ cd directorio    # cambiar directorio
$ mkdir data       # crear carpeta
$ touch notas.txt  # archivo vacío
$ cat notas.txt    # ver contenido
$ rm notas.txt     # borrar
$ man ls           # ayuda
\end{lstlisting}
\end{column}
\begin{column}{0.4\textwidth}
% TODO: imagen terminal
\includegraphics[width=\linewidth]{example-image-b}
\end{column}
\end{columns}
\end{frame}

%---------------- permisos -------------------------
\begin{frame}[fragile]{Permisos de archivos}
\begin{lstlisting}[style=shell]
$ ls -l backup.sh
-rwxr-xr-- 1 pi pi  54 jul  8 12:30 backup.sh
\end{lstlisting}
\begin{itemize}
  \item \texttt{rwx}: permisos del propietario
  \item \texttt{r-x}: grupo
  \item \texttt{r--}: otros
\end{itemize}
\pause
\textbf{Cambiar permisos / dueño}
\begin{lstlisting}[style=shell]
$ chmod u+x backup.sh        # añadir ejec.
$ sudo chown root:root file  # cambiar propietario
\end{lstlisting}
\end{frame}

%---------------- variables & echo -----------------
\begin{frame}{Variables y \texttt{echo}}
\begin{lstlisting}[style=shell]
$ NODENAME=$(hostname)
$ echo "Hola, estoy en $NODENAME"
\end{lstlisting}
\pause
\textbf{Redirección y pipes}
\begin{lstlisting}[style=shell]
$ ls -l /etc > listado.txt   # salida a archivo
$ cat listado.txt | less     # pipe
\end{lstlisting}
\end{frame}

%---------------- actividad ------------------------
\begin{frame}{Actividad práctica (30 min)}
\begin{enumerate}
  \item Crear la estructura:\newline
        \texttt{~/proyectos/\{scripts,datos,logs\}}
  \item Escribir \texttt{backup.sh} que:
    \begin{itemize}
      \item Reciba nombre de carpeta (\$1)
      \item Comprima con \texttt{tar} y genere \texttt{.tar.gz}
      \item Mueva el archivo a \texttt{~/backups}
    \end{itemize}
  \item Añadir bit de ejecución y probarlo.
\end{enumerate}
\pause
\small\emoji{bulb} Tip: usa \texttt{\$(date +\%F)} para sellar con la fecha.
\end{frame}

%---------------- cheatsheet -----------------------
\begin{frame}{Mini cheatsheet}
\begin{columns}[T]
\begin{column}{0.48\textwidth}
\begin{itemize}
  \item \textbf{Ver procesos}: \texttt{htop}, \texttt{ps aux}
  \item \textbf{Uso de disco}: \texttt{df -h}, \texttt{du -sh *}
  \item \textbf{Búsqueda}: \texttt{grep}, \texttt{find}
  \item \textbf{Red}: \texttt{ip a}, \texttt{ping}, \texttt{hostname -I}
\end{itemize}
\end{column}
\begin{column}{0.48\textwidth}
\begin{itemize}
  \item \textbf{Editar archivos}: \texttt{nano}, \texttt{vim}
  \item \textbf{Actualizar apt}: \texttt{sudo apt update && apt upgrade}
  \item \textbf{Historial}: \texttt{history}, \texttt{!42}
  \item \textbf{Autocompletar}: \texttt{TAB}
\end{itemize}
\end{column}
\end{columns}
\end{frame}

%---------------- recursos -------------------------
\begin{frame}{Recursos recomendados}
\begin{itemize}
  \item «The Linux Command Line», W. Shotts Jr. (gratuito).
  \item https://linuxjourney.com (interactivo).
  \item Documentación oficial Raspberry Pi OS.
\end{itemize}
\end{frame}

%---------------- cierre ---------------------------
\begin{frame}[standout]
  ¡Buen trabajo!\\
  Próximo bloque: redes IP y claves SSH
\end{frame}

\end{document}

