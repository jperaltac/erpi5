% !TEX program = lualatex
\documentclass[aspectratio=169, professionalfonts]{beamer}
\usepackage[spanish]{babel}
\usepackage{fontspec}
\usepackage{graphicx}
\usepackage{mwe}            % imágenes de ejemplo
\usepackage{xcolor}
\usepackage{emoji}
\setmainfont{Libertinus Serif}

\title[ClústerLab • Día 1]{ClústerLab 2025\\\huge Introducción y Motivación}
\subtitle{Escuela de Computación Distribuida – ITM Medellín}
\author{Equipo docente ClústerLab}
\date{4 de agosto 2025}

%--------------------------------------------------------------------
\begin{document}

%------------------------ title -------------------------------------
\begin{frame}[plain]
  \titlepage
\end{frame}

%------------------------ objetivos ---------------------------------
\begin{frame}{Objetivos del curso}
  \begin{itemize}
    \item ¿Qué es un \emph{clúster}? ¿Por qué importa?
    \item Configurar un mini-clúster con Raspberry Pi 5.
    \item Fundamentos de Linux, scripting Bash y redes.
    \item Programación paralela (MPI) y proyecto final.
  \end{itemize}
\end{frame}

%------------------------ qué es un clúster -------------------------
\begin{frame}{¿Qué es un clúster?}
  \begin{columns}[T]
    \begin{column}{0.55\textwidth}
      \begin{block}{Definición rápida}
        Conjunto de computadores interconectados que actúan
        como un único recurso de cómputo.
      \end{block}
      \pause
      \begin{itemize}
        \item Alta disponibilidad y escalabilidad \emoji{chart-increasing}
        \item Paralelismo y balanceo de carga
        \item Ciencia de datos, simulaciones, IA …
      \end{itemize}
    \end{column}
    \begin{column}{0.4\textwidth}
      \centering
      \includegraphics[width=\linewidth]{example-image} % imagen genérica
    \end{column}
  \end{columns}
\end{frame}

%------------------------ por qué raspberry -------------------------
\begin{frame}{¿Por qué Raspberry Pi 5?}
  \begin{itemize}
    \item Bajo costo y consumo $\approx$ 5–7 W.
    \item Soporte completo Linux + comunidad enorme.
    \item Perfecta para aprender HPC en pequeño.
    \item ¡Te la llevas a casa! \emoji{rocket}
  \end{itemize}
  \vspace{0.4em}
  \centering
  \includegraphics[width=0.7\linewidth]{example-image-duck} % otra imagen demo
\end{frame}

%------------------------ agenda del día ----------------------------
\begin{frame}{Agenda Día 1 (09:00–13:00)}
  \begin{enumerate}
    \item 09:00 – 09:15  Bienvenida y motivación.
    \item 09:15 – 10:30  Encender la Pi + primeros comandos Linux.
    \item 10:45 – 12:00  Bash I (variables, redirección).
    \item 12:00 – 13:00  Actividad: árbol de carpetas + backup.
  \end{enumerate}
\end{frame}

%------------------------ requisitos -------------------------------
\begin{frame}{Requisitos para hoy}
  \begin{itemize}
    \item Laptop con Wi-Fi y cliente SSH (Linux/macOS) \textbf{o} WSL 2 en Windows.
    \item Raspberry Pi 5 con SD Card (≥16 GB) y fuente.
    \item Cable LAN y acceso al switch del aula.
  \end{itemize}
  \alert{En Windows usa \textbf{WSL 2 + Ubuntu}.}
\end{frame}

%------------------------ WSL --------------------------------------
\begin{frame}{WSL 2 en Windows (instalación rápida)}
  \begin{enumerate}
    \item Activar «Sub-sistema Windows para Linux» y «Plataforma VM».
    \item En PowerShell (Admin): \texttt{wsl --install Ubuntu}
    \item Reiniciar, abrir Ubuntu WSL y crear usuario.
    \item Instalar \texttt{openssh-client}:\newline
          \texttt{\$ sudo apt install openssh-client}
  \end{enumerate}
  \pause
  \begin{block}{Conexión SSH (ejemplo)}
\texttt{\$ ssh pi@192.168.1.50}
  \end{block}
\end{frame}

%------------------------ kit ------------------------
\begin{frame}{Material necesario (checklist)}
  \begin{columns}[T]
    \begin{column}{0.48\textwidth}
      \begin{itemize}
        \item Raspberry Pi 5 + disipador
        \item Tarjeta µSD con Raspberry Pi OS Lite
        \item Fuente USB-C (5 V ≥ 3 A)
        \item Cable LAN Cat 5e/6
      \end{itemize}
    \end{column}
    \begin{column}{0.48\textwidth}
      \includegraphics[width=\linewidth]{example-image-b}
    \end{column}
  \end{columns}
\end{frame}

%------------------------ cierre -----------------------------------
\begin{frame}%[standout]
  ¡Comencemos!\\
  \small Siguiente: encender la Pi y usar la terminal
\end{frame}

\end{document}

