% !TEX program = lualatex
\documentclass[aspectratio=169,professionalfonts]{beamer}
%------------------------- Preamble (beamer) -------------------------%
%   Require LuaLaTeX + --shell-escape (for minted)                    %
%--------------------------------------------------------------------%
\usepackage[spanish]{babel}

% ---------- Fonts (modern look) ----------
\usepackage{fontspec}
\defaultfontfeatures{Ligatures=TeX, Scale=MatchLowercase}
\setmainfont{Inter}       % Sans‑serif body (install ttf–Inter)
\setsansfont{Inter}
\setmonofont{Fira Code}   % Nice mono for code

% ---------- Theme ----------
\usetheme[
  titleformat=smallcaps,
  numbering=fraction,
  progressbar=frametitle   % thin progress line under the title
]{metropolis}              % modern beamer theme
\metroset{block=fill}      % coloured blocks

% ---------- Colours ----------
\definecolor{clblue}{HTML}{005F9E}
\definecolor{clorange}{HTML}{FF9130}
\definecolor{clgrey}{HTML}{F4F4F4}

% ---------- minted ----------
\usepackage{minted}
\setminted{
  fontsize=\footnotesize,
  autogobble,
  breaklines,
  tabsize=2,
  style=monokai,
  linenos
}

% ---------- Custom boxes (tcolorbox) ----------
\usepackage{tcolorbox}
\tcbuselibrary{skins, breakable}
% Warning box
\newtcolorbox{warnbox}{
  colback=clorange!10,
  colframe=clorange!80!black,
  breakable,
  title={\faWarning\quad Advertencia},
  fonttitle=\bfseries,
  left=2mm, right=2mm, top=1mm, bottom=1mm
}
% Information box
\newtcolorbox{infobox}{
  colback=clblue!5,
  colframe=clblue!60!black,
  breakable,
  title={\faInfoCircle\quad Nota},
  fonttitle=\bfseries,
  left=2mm, right=2mm, top=1mm, bottom=1mm
}

% ---------- Icons ----------
\usepackage{fontawesome5}

% ---------- Misc tweaks ----------
\graphicspath{{figures/}}
\setlength{\parskip}{0.5em}
\addto\extrasspanish{\renewcommand{\contentsname}{Índice}}
%--------------------------------------------------------------------%


\usepackage{mwe} % figuras de ejemplo -> example-image

\title[ClústerLab • Día 2]{Redes IP y claves SSH}
\subtitle{Configurar IP estática y generar claves}
\author{Equipo docente ClústerLab}
\date{6 de agosto de 2025}

\begin{document}

%------------------------------ portada ----------------------------
\begin{frame}[plain]
  \titlepage
\end{frame}

%------------------------------ ssh test ---------------------------
\begin{frame}[fragile]{Probar acceso SSH}
  \begin{minted}{bash}
$ ssh pi@10.0.0.21 hostname
  \end{minted}
  Si la autenticación funciona, verás el nombre de tu Pi en la salida.
  \begin{minted}{bash}
# Opcional en ~/.ssh/config
Host rpi21
  HostName 10.0.0.21
  User pi
  \end{minted}
  Entonces basta con ejecutar:
  \begin{minted}{bash}
$ ssh rpi21
  \end{minted}
\end{frame}

%------------------------------ nmap opciones ----------------------
\begin{frame}[fragile]{Opciones útiles de \texttt{nmap}}
  \begin{minted}{bash}
$ nmap -A 10.0.0.0/24       # detección de servicios y SO
$ nmap --open -p 80 10.0.0.* # buscar servidores web
  \end{minted}
  \begin{itemize}
    \item \texttt{-A} entrega información detallada y requiere más tiempo.
    \item Limita el rango si la red está muy concurrida.
  \end{itemize}
\end{frame}

%------------------------------ prueba ip -------------------------
\begin{frame}[fragile]{Verificar conectividad}
  \begin{minted}{bash}
$ ip addr show eth0 | grep 'inet '
$ ping -c3 10.0.0.1
  \end{minted}
  \begin{itemize}
    \item Confirma que la IP asignada se muestre en \texttt{inet}.
    \item Un par de \texttt{ping} verifica la ruta al gateway.
  \end{itemize}
  \begin{center}
    \includegraphics[width=.45\textwidth]{example-image-a}
  \end{center}
\end{frame}

%------------------------------ objetivos --------------------------
\begin{frame}[fragile]{Objetivos de la sesión}
  \begin{itemize}
    \item Repasar conceptos básicos de redes IP y asignar una dirección estática.
    \item Descubrir vecinos en la subred usando \texttt{nmap}.
    \item Generar y copiar una clave SSH para acceso sin contraseña.
    \item Preparar la Pi para trabajo remoto desde la siguiente sesión.
  \end{itemize}
\end{frame}

%------------------------------ agenda -----------------------------
\begin{frame}[fragile]{Agenda (09:00 – 10:30)}
  \begin{enumerate}
    \item Revisión flash de SD/SSH (5 preguntas tipo Kahoot).
    \item Redes IP y configuración estática via \texttt{dhcpcd.conf}.
    \item Escaneo de vecinos con \texttt{nmap}.
    \item Generar par de claves RSA y copiar a la Raspberry Pi.
  \end{enumerate}
\end{frame}

%------------------------------ revisión flash ---------------------
\begin{frame}[fragile]{Revisión flash}
  \begin{itemize}
    \item Cinco preguntas rápidas para reforzar conceptos del Día 1.
    \item Utiliza tu teléfono o laptop para responder en tiempo real.
    \item Comparte dudas antes de avanzar a redes.
  \end{itemize}
\end{frame}

%------------------------------ IP estática ------------------------
\begin{frame}[fragile]{Configurar IP estática}
  Edita \texttt{/etc/dhcpcd.conf} para definir una IP fija:
  \begin{minted}{bash}
sudo nano /etc/dhcpcd.conf
interface eth0
static ip_address=10.0.0.21/24
static routers=10.0.0.1
static domain_name_servers=8.8.8.8 8.8.4.4
  \end{minted}
  Después de guardar, reinicia el servicio:
  \begin{minted}{bash}
$ sudo systemctl restart dhcpcd
  \end{minted}
  \begin{infobox}
  Elige una dirección única dentro de la subred del aula (10.0.0.X). Consulta con tu grupo para evitar conflictos.
  \end{infobox}
\end{frame}

%------------------------------ nmap scan --------------------------
\begin{frame}[fragile]{Descubrir vecinos con \texttt{nmap}}
  Instala la herramienta y explora quién está conectado:
  \begin{minted}{bash}
$ sudo apt install -y nmap
$ nmap -sn 10.0.0.0/24          # escaneo rápido de hosts
$ nmap -p 22 --open 10.0.0.0/24 # encontrar servicios SSH activos
  \end{minted}
  \begin{itemize}
    \item La opción \texttt{-sn} (ping scan) lista IP y latencias.
    \item Usa \texttt{-p} para verificar puertos específicos como SSH.
  \end{itemize}
\end{frame}

%------------------------------ claves SSH -------------------------
\begin{frame}[fragile]{Generar y copiar una clave SSH}
  \begin{minted}{bash}
# Genera un par RSA de 4096 bits con comentario
$ ssh-keygen -t rsa -b 4096 -C "tu@correo"
# Copia la clave pública a la Raspberry Pi
$ ssh-copy-id pi@10.0.0.21
  \end{minted}
  \begin{infobox}
  La autenticación sin contraseña simplifica el uso de \texttt{scp} y \texttt{rsync}. Protege tu clave privada (\texttt{~/.ssh/id\_rsa}) con permisos 600.
  \end{infobox}
\end{frame}

%------------------------------ actividad --------------------------
\begin{frame}[fragile]{Actividad práctica}
  \begin{enumerate}
    \item Asigna una IP estática a tu Pi y reinicia la interfaz.
    \item Escanea la red para registrar IP y hostname de tus compañeros.
    \item Genera tu par de claves y copia la pública a la Pi.
    \item Conéctate vía SSH y ejecuta \texttt{uptime} para validar acceso.
    \item Comparte la lista de IP/hostnames en el canal del curso.
  \end{enumerate}
\end{frame}

%------------------------------ resumen ----------------------------
\begin{frame}[fragile]{Resumen}
  \begin{itemize}
    \item Configuraste una IP estática con \texttt{dhcpcd.conf} y reiniciaste el servicio.
    \item Descubriste dispositivos en tu subred usando \texttt{nmap}.
    \item Generaste una clave SSH segura y habilitaste el acceso sin contraseña.
  \end{itemize}
  \vspace{0.5em}
  Listo para comenzar a trabajar de manera remota a partir del Día 3.
\end{frame}

\end{document}