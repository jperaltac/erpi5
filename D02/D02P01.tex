% !TEX program = lualatex
\documentclass[aspectratio=169, professionalfonts]{beamer}
\usepackage[spanish]{babel}
\usepackage{fontspec}
\usepackage{graphicx}
\usepackage{verbatim}
\usepackage{mwe}            % imágenes de ejemplo
\usepackage{xcolor}
\usepackage{emoji}
\usepackage{minted}         % bloques de código
\setminted{
  fontsize=\small,
  bgcolor=gray!10,
  linenos=false
}
\setmainfont{Libertinus Serif}

\title[ClústerLab • Día 2]{ClústerLab 2025\\\huge Redes IP y SSH seguro}
\subtitle{Escuela de Computación Distribuida – ITM Medellín}
\author{Equipo docente ClústerLab}
\date{6 de agosto 2025}

\begin{document}

%---------------------------------- title ----------------------------------
\begin{frame}[plain]
  \titlepage
\end{frame}

%---------------------------------- agenda ---------------------------------
\begin{frame}{Agenda Día 2 (09:00–13:00)}
\begin{enumerate}
  \item 09:00–09:15  Revisión rápida (comandos Día 1)
  \item 09:15–10:30  Conceptos IP, máscaras y subredes
  \item 10:45–11:30  SSH seguro + claves públicas
  \item 11:30–12:00  Transferencia y montaje remoto (\texttt{scp}/\texttt{sshfs})
  \item 12:00–13:00  Actividad: \texttt{ping\_all.sh} + compartir carpeta
\end{enumerate}
\end{frame}

%---------------------------------- conceptos IP ---------------------------
\begin{frame}{Dirección IP vs. máscara de red}
\begin{columns}[T]
\begin{column}{0.55\textwidth}
\begin{block}{IPv4}
  \texttt{192.168.\textbf{1}.42}\newline
  \small 4 octetos = 32 bits
\end{block}
\pause
\begin{itemize}
  \item Red casera típica → \texttt{192.168.1.0/24}
  \item Máscara \texttt{255.255.255.0} (24 bits de red)
  \item Último octeto identifica el host (0–255)
\end{itemize}
\end{column}
\begin{column}{0.4\textwidth}
  \centering\includegraphics[width=\linewidth]{example-image}
\end{column}
\end{columns}
\end{frame}

%---------------------------------- comandos red ---------------------------
\begin{frame}{Comandos de red útiles}
\begin{itemize}
  \item Ver interfaces → \mintinline{bash}|ip a|
  \item Tabla de rutas → \mintinline{bash}|ip r|
  \item Comprobar conectividad → \mintinline{bash}|ping 192.168.1.1|
  \item Escanear red → \mintinline{bash}|nmap -sP 192.168.1.0/24|
\end{itemize}
\pause
\begin{block}{WSL tip}
En WSL 2 las subredes son distintas; usa la IP de tu adaptador físico (Ethernet/Wi-Fi) para conectarte a las Pi.
\end{block}
\end{frame}

%---------------------------------- ssh seguro -----------------------------
\begin{frame}[fragile]{SSH con clave pública}
\begin{enumerate}
  \item Generar clave (local):  
\begin{minted}{bash}
$ ssh-keygen -t ed25519
\end{minted}
  \item Copiar clave a la Pi:  
\begin{minted}{bash}
$ ssh-copy-id pi@192.168.1.50
\end{minted}
  \item Probar login sin contraseña:  
\begin{minted}{bash}
$ ssh pi@192.168.1.50
\end{minted}
\end{enumerate}
\pause
\small
\emoji{lock} Para deshabilitar contraseñas: editar \texttt{/etc/ssh/sshd\_config} y poner  
\texttt{PasswordAuthentication no}, luego \texttt{sudo systemctl restart ssh}.
\end{frame}


\begin{comment}

\begin{frame}{Copiar y montar carpetas remotas}
\begin{columns}[T]
\begin{column}{0.48\textwidth}
\textbf{\texttt{scp}}
\begin{minted}{bash}
$ scp datos.csv pi@192.168.1.50:/home/pi/
\end{minted}
\end{column}
\begin{column}{0.48\textwidth}
\textbf{\texttt{sshfs}}
%\begin{minted}{bash}
%$ mkdir ~/pi_home
%$ sshfs pi@192.168.1.50:/home/pi ~/pi\_home
%\end{minted}
\end{column}
\end{columns}
\pause
\centering
\includegraphics[width=0.7\linewidth]{example-image-duck}
\end{frame}

\end{comment}

%---------------------------------- actividad ------------------------------
\begin{comment}
\begin{frame}{Actividad (12:00 – 13:00)}
\begin{enumerate}
  \item Escribe \texttt{ping\_all.sh}:  
\begin{minted}{bash}
#!/usr/bin/env bash
for ip in 192.168.1.{50..60}; do
  if ping -c1 -W1 $ip &>/dev/null; then
    echo "$ip OK"
  else
    echo "$ip FAIL"
  fi
done
\end{minted}
  \item Colócalo en \texttt{\/home\/pi\/bin}, hazlo ejecutable y pruébalo.  
  \item Monta una carpeta remota vía \texttt{sshfs} y transfiere un archivo.  
\end{enumerate}
\end{frame}
\end{comment}

\begin{comment}
\begin{frame}[standout]
Red y SSH listos\\
\small Siguiente Bash avanzado y automatización
\end{frame}
\end{comment}

\end{document}

