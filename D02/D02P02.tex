% !TEX program = lualatex
\documentclass[aspectratio=169,professionalfonts]{beamer}
%------------------------- Preamble (beamer) -------------------------%
%   Require LuaLaTeX + --shell-escape (for minted)                    %
%--------------------------------------------------------------------%
\usepackage[spanish]{babel}

% ---------- Fonts (modern look) ----------
\usepackage{fontspec}
\defaultfontfeatures{Ligatures=TeX, Scale=MatchLowercase}
\setmainfont{Inter}       % Sans‑serif body (install ttf–Inter)
\setsansfont{Inter}
\setmonofont{Fira Code}   % Nice mono for code

% ---------- Theme ----------
\usetheme[
  titleformat=smallcaps,
  numbering=fraction,
  progressbar=frametitle   % thin progress line under the title
]{metropolis}              % modern beamer theme
\metroset{block=fill}      % coloured blocks

% ---------- Colours ----------
\definecolor{clblue}{HTML}{005F9E}
\definecolor{clorange}{HTML}{FF9130}
\definecolor{clgrey}{HTML}{F4F4F4}

% ---------- minted ----------
\usepackage{minted}
\setminted{
  fontsize=\footnotesize,
  autogobble,
  breaklines,
  tabsize=2,
  style=monokai,
  linenos
}

% ---------- Custom boxes (tcolorbox) ----------
\usepackage{tcolorbox}
\tcbuselibrary{skins, breakable}
% Warning box
\newtcolorbox{warnbox}{
  colback=clorange!10,
  colframe=clorange!80!black,
  breakable,
  title={\faWarning\quad Advertencia},
  fonttitle=\bfseries,
  left=2mm, right=2mm, top=1mm, bottom=1mm
}
% Information box
\newtcolorbox{infobox}{
  colback=clblue!5,
  colframe=clblue!60!black,
  breakable,
  title={\faInfoCircle\quad Nota},
  fonttitle=\bfseries,
  left=2mm, right=2mm, top=1mm, bottom=1mm
}

% ---------- Icons ----------
\usepackage{fontawesome5}

% ---------- Misc tweaks ----------
\graphicspath{{figures/}}
\setlength{\parskip}{0.5em}
\addto\extrasspanish{\renewcommand{\contentsname}{Índice}}
%--------------------------------------------------------------------%



\title[ClústerLab • Día 2]{Bash avanzado y transferencia}
\subtitle{Condicionales, bucles y copias seguras}
\author{Equipo docente ClústerLab}
\date{6 de agosto de 2025}

\begin{document}

%------------------------------ portada ----------------------------
\begin{frame}[plain]
  \titlepage
\end{frame}

%------------------------------ objetivos --------------------------
\begin{frame}[fragile]{Objetivos de la sesión}
  \begin{itemize}
    \item Comprender condicionales \texttt{if}/\texttt{elif}/\texttt{else} en Bash.
    \item Utilizar bucles \texttt{for}, \texttt{while} y \texttt{case}.
    \item Desarrollar \texttt{ping\_all.sh} para medir latencias en la subred.
    \item Transferir archivos con \texttt{scp} y \texttt{rsync}, verificando integridad con \texttt{sha256sum}.
  \end{itemize}
\end{frame}

%------------------------------ agenda -----------------------------
\begin{frame}[fragile]{Agenda (10:45 – 13:00)}
  \begin{enumerate}
    \item Condicionales en Bash: \texttt{if}, \texttt{elif}, \texttt{else}.
    \item Bucles \texttt{for}, \texttt{while} y estructura \texttt{case}.
    \item Script \texttt{ping\_all.sh} para recorrer una subred.
    \item Copiar y sincronizar datos con \texttt{scp}/\texttt{rsync}.
    \item Verificar archivos con \texttt{sha256sum}.
  \end{enumerate}
\end{frame}

%------------------------------ condicionales ----------------------
\begin{frame}[fragile]{Condicionales en Bash}
  Evaluar condiciones y ejecutar bloques de código:
  \begin{minted}{bash}
#!/usr/bin/env bash
FILE="$1"
if [ -f "$FILE" ]; then
  echo "El archivo $FILE existe"
elif [ -d "$FILE" ]; then
  echo "$FILE es un directorio"
else
  echo "No existe $FILE"
fi
  \end{minted}
  \begin{infobox}
  Las expresiones entre corchetes \texttt{[ ]} admiten comparaciones de cadenas (\texttt{=}, \texttt{!=}) y números (\texttt{-lt}, \texttt{-eq}, etc.).
  \end{infobox}
\end{frame}

%------------------------------ bucles -----------------------------
\begin{frame}[fragile]{Bucles \texttt{for} y \texttt{while}}
  \begin{minted}{bash}
# Recorre direcciones IP de 1 a 254
for host in {1..254}; do
  ip="10.0.0.$host"
  ping -c1 -W1 $ip &> /dev/null && echo "$ip activo"
done

# Cuenta de 0 a 4 con while
i=0
while [ $i -lt 5 ]; do
  echo "Número $i"
  ((i++))
done
  \end{minted}
  \begin{infobox}
  Utiliza \texttt{case} para múltiples posibilidades:
  \begin{minted}{bash}
case "$1" in
  start) echo "Arrancando";;
  stop)  echo "Deteniendo";;
  *)     echo "Uso: $0 {start|stop}";;
esac
  \end{minted}
  \end{infobox}
\end{frame}

%------------------------------ ping_all ---------------------------
\begin{frame}[fragile]{Script \texttt{ping\_all.sh}}
  \begin{minted}{bash}
#!/usr/bin/env bash
network="10.0.0"
for host in {1..254}; do
  ip="$network.$host"
  if ping -c1 -W1 "$ip" &> /dev/null; then
    rtt=$(ping -c1 -W1 "$ip" | grep -oP 'time=\K[\d.]+')
    echo "$ip responde en ${rtt} ms"
  fi
done
  \end{minted}
  \begin{warnbox}
  Ejecutar el script en paralelo con varios grupos puede generar tráfico intenso; coordina con tu instructor.
  \end{warnbox}
\end{frame}

%------------------------------ transferencia ---------------------
\begin{frame}[fragile]{Transferencia con \texttt{scp} y \texttt{rsync}}
  Copia un archivo a la Pi:
  \begin{minted}{bash}
$ scp informe.pdf pi@10.0.0.21:~/
  \end{minted}
  Sincroniza un directorio local con el remoto:
  \begin{minted}{bash}
$ rsync -avz proyectos/ pi@10.0.0.21:~/proyectos/
  \end{minted}
  \begin{infobox}
  La opción \texttt{-z} de \texttt{rsync} habilita compresión; \texttt{-a} conserva permisos y tiempos.
  \end{infobox}
\end{frame}

%------------------------------ hash -------------------------------
\begin{frame}[fragile]{Verificación con \texttt{sha256sum}}
  Calcula y compara hashes para asegurar integridad:
  \begin{minted}{bash}
$ sha256sum archivo.txt
e3b0c44298fc1c149afbf4c8996fb92427ae41e4649b934ca495991b7852b855  archivo.txt
  \end{minted}
  \begin{itemize}
    \item Ejecuta el comando antes y después de transferir.
    \item Si los hashes coinciden, la copia es correcta.
  \end{itemize}
\end{frame}

%------------------------------ actividad --------------------------
\begin{frame}[fragile]{Actividad práctica}
  \begin{enumerate}
    \item Implementa y ejecuta \texttt{ping\_all.sh} en tu subred.
    \item Copia un archivo de al menos 10 MB a tu Pi usando \texttt{scp} y mide el tiempo.
    \item Repite la copia con \texttt{rsync} y compara tiempos.
    \item Verifica los hashes SHA‑256 para confirmar integridad.
  \end{enumerate}
\end{frame}

%------------------------------ resumen ----------------------------
\begin{frame}[fragile]{Resumen}
  \begin{itemize}
    \item Aprendiste a escribir condicionales y bucles en Bash.
    \item Creaste un script para mapear y medir latencias en la red.
    \item Transferiste y sincronizaste archivos de forma eficiente y segura.
  \end{itemize}
  \vspace{0.5em}
  ¡Terminó la configuración básica de tu entorno remoto!
\end{frame}

\end{document}