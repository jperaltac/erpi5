% !TEX program = lualatex
\documentclass[aspectratio=169, professionalfonts]{beamer}
\usepackage[spanish]{babel}
\usepackage{fontspec}
\usepackage{graphicx}
\usepackage{mwe}
\usepackage{xcolor}
\usepackage{minted}
\usepackage{emoji}
\setminted{
  fontsize=\small,
  bgcolor=gray!10,
  linenos=false
}
\setmainfont{Libertinus Serif}

\title[ClústerLab • Día 2]{Bash avanzado: condicionales y bucles}
\subtitle{Hands-on en Raspberry Pi 5}
\author{Equipo docente ClústerLab}
\date{6 de agosto 2025}

\begin{document}

%---------------------------------- title ----------------------------------
\begin{frame}[plain]
  \titlepage
\end{frame}

%---------------------------------- if -------------------------------------
\begin{frame}[fragile]{Condicional \texttt{if}}
\begin{minted}{bash}
read -p "Archivo: " f
if [[ -f $f ]]; then
  echo "Existe y es regular"
else
  echo "No existe"
fi
\end{minted}
\pause
\begin{itemize}
  \item \texttt{-f}, \texttt{-d}, \texttt{-x}, \texttt{-s}  pruebas de archivo
  %\item Comparar cadenas: \texttt{[[ "\$var" == "abc" ]]}
  %\item Código de salida en \texttt{\$?}
\end{itemize}
\end{frame}

%---------------------------------- for/while --------------------------------
\begin{frame}[fragile]{Bucles \texttt{for} y \texttt{while}}
\begin{columns}[T]
\begin{column}{0.55\textwidth}
\textbf{\texttt{for}}
\begin{minted}{bash}
for n in {1..5}; do
  echo "Iteración $n"
done
\end{minted}

\textbf{\texttt{while}}
\begin{minted}{bash}
contador=0
while (( contador < 3 )); do
  echo $contador
  ((contador++))
done
\end{minted}
\end{column}
\begin{column}{0.4\textwidth}
\includegraphics[width=\linewidth]{example-image-b}
\end{column}
\end{columns}
\end{frame}

%---------------------------------- case -----------------------------------
\begin{frame}[fragile]{Estructura \texttt{case}}
\begin{minted}{bash}
case $1 in
  start)  systemctl start myapp ;;
  stop)   systemctl stop  myapp ;;
  status) systemctl status myapp ;;
  *)      echo "Uso: $0 {start|stop|status}" ;;
esac
\end{minted}
\end{frame}

%---------------------------------- arrays ----------------------------------
\begin{frame}[fragile]{Arrays en Bash}
\begin{minted}{bash}
HOSTS=(node1 node2 node3)
echo "Total: ${#HOSTS[@]}"
for h in "${HOSTS[@]}"; do
  ssh $h uptime
done
\end{minted}
\end{frame}

%---------------------------------- script completo -------------------------
\begin{frame}[fragile]{Ejemplo completo — \texttt{ping\_all.sh}}
\begin{minted}{bash}
#!/usr/bin/env bash
IPS=(192.168.1.{50..60})
for ip in "${IPS[@]}"; do
  if ping -c1 -W1 $ip &>/dev/null; then
    echo -e "$ip \e[32mOK\e[0m"
  else
    echo -e "$ip \e[31mFAIL\e[0m"
  fi
done
\end{minted}
\end{frame}

%---------------------------------- sshfs -----------------------------------
\begin{frame}[fragile]{Montar carpeta remota con \texttt{sshfs}}
\begin{minted}{bash}
$ sudo apt install sshfs         # si es necesario
$ mkdir ~/pi_share
$ sshfs pi@node2:/home/pi ~/pi_share
...
$ fusermount -u ~/pi_share       # desmontar
\end{minted}
\end{frame}

%---------------------------------- reto ------------------------------------
\begin{frame}[fragile]{Mini-reto (15 min)}
\begin{enumerate}
  %\item Modifica \texttt{ping\_all.sh} para guardar salida en \texttt{logs/\$(date +\%F).txt}.  
  \item Copia el log a otra Pi con \texttt{scp}.  
  \item Verifica permisos y propietario después de la transferencia.  
\end{enumerate}
\end{frame}

%---------------------------------- recursos --------------------------------
\begin{frame}{Recursos extra}
\begin{itemize}
  \item Advanced Bash Scripting Guide — TLDP  
  \item \url{https://explainshell.com}  
  \item \texttt{man bash}, \texttt{help builtin}  
\end{itemize}
\end{frame}

%---------------------------------- cierre ----------------------------------
\begin{frame}[fragile]
¡Trabajo completado!\\
\small Próximo día → Raspberry Pi OS y monitoreo
\end{frame}

\end{document}

