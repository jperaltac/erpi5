% !TEX program = lualatex
\documentclass[aspectratio=169,professionalfonts]{beamer}
%------------------------- Preamble (beamer) -------------------------%
%   Require LuaLaTeX + --shell-escape (for minted)                    %
%--------------------------------------------------------------------%
\usepackage[spanish]{babel}

% ---------- Fonts (modern look) ----------
\usepackage{fontspec}
\defaultfontfeatures{Ligatures=TeX, Scale=MatchLowercase}
\setmainfont{Inter}       % Sans‑serif body (install ttf–Inter)
\setsansfont{Inter}
\setmonofont{Fira Code}   % Nice mono for code

% ---------- Theme ----------
\usetheme[
  titleformat=smallcaps,
  numbering=fraction,
  progressbar=frametitle   % thin progress line under the title
]{metropolis}              % modern beamer theme
\metroset{block=fill}      % coloured blocks

% ---------- Colours ----------
\definecolor{clblue}{HTML}{005F9E}
\definecolor{clorange}{HTML}{FF9130}
\definecolor{clgrey}{HTML}{F4F4F4}

% ---------- minted ----------
\usepackage{minted}
\setminted{
  fontsize=\footnotesize,
  autogobble,
  breaklines,
  tabsize=2,
  style=monokai,
  linenos
}

% ---------- Custom boxes (tcolorbox) ----------
\usepackage{tcolorbox}
\tcbuselibrary{skins, breakable}
% Warning box
\newtcolorbox{warnbox}{
  colback=clorange!10,
  colframe=clorange!80!black,
  breakable,
  title={\faWarning\quad Advertencia},
  fonttitle=\bfseries,
  left=2mm, right=2mm, top=1mm, bottom=1mm
}
% Information box
\newtcolorbox{infobox}{
  colback=clblue!5,
  colframe=clblue!60!black,
  breakable,
  title={\faInfoCircle\quad Nota},
  fonttitle=\bfseries,
  left=2mm, right=2mm, top=1mm, bottom=1mm
}

% ---------- Icons ----------
\usepackage{fontawesome5}

% ---------- Misc tweaks ----------
\graphicspath{{figures/}}
\setlength{\parskip}{0.5em}
\addto\extrasspanish{\renewcommand{\contentsname}{Índice}}
%--------------------------------------------------------------------%



\title[ClústerLab • Día 5]{Instalación de Slurm}
\subtitle{Controlador y demonios de cómputo}
\author{Equipo docente ClústerLab}
\date{13 de agosto de 2025}

\begin{document}

%------------------------------ portada ----------------------------
\begin{frame}[plain]
  \titlepage
\end{frame}

%------------------------------ objetivos --------------------------
\begin{frame}[fragile]{Objetivos de la sesión}
  \begin{itemize}
    \item Instalar \texttt{slurmctld} (controlador) en el nodo head.
    \item Instalar \texttt{slurmd} (demonio) en cada nodo de cálculo.
    \item Generar y ajustar un \texttt{slurm.conf} mínimo para la Pi‑farm.
    \item Verificar el estado de los nodos con \texttt{sinfo}.
  \end{itemize}
\end{frame}

%------------------------------ agenda -----------------------------
\begin{frame}[fragile]{Agenda (09:00 – 11:00)}
  \begin{enumerate}
    \item Paquetes necesarios y repositorios.
    \item Configurar usuario \texttt{slurm} y directorios de estado.
    \item Crear \texttt{slurm.conf} con la herramienta \texttt{configurator.html}.
    \item Sincronizar la configuración en todos los nodos.
    \item Comprobación con \texttt{scontrol} y \texttt{sinfo}.
    \item Actividad práctica.
  \end{enumerate}
\end{frame}

%------------------------------ install head -----------------------
\begin{frame}[fragile]{Instalación en el nodo head}
  \begin{minted}{bash}
$ sudo apt install slurmctld slurmdbd munge
$ sudo systemctl enable --now munge
$ sudo systemctl enable slurmctld
  \end{minted}
  \begin{itemize}
    \item \textbf{Munge} proporciona autenticación.
    \item Copiar la clave \texttt{/etc/munge/munge.key} a los nodos.
  \end{itemize}
\end{frame}

%------------------------------ install node -----------------------
\begin{frame}[fragile]{Instalación en nodos de cálculo}
  \begin{minted}{bash}
$ sudo apt install slurmd munge
$ sudo systemctl enable --now munge
$ sudo systemctl enable slurmd
  \end{minted}
  Asegúrate de que la hora esté sincronizada (\texttt{chrony} o \texttt{systemd‑timesyncd}).
\end{frame}

%------------------------------ slurm.conf -------------------------
\begin{frame}[fragile]{Ejemplo mínimo de \texttt{slurm.conf}}
  \begin{minted}[bgcolor=gray!15]{ini}
ClusterName=PiCluster
SlurmctldHost=head
MpiDefault=pmix
ProctrackType=proctrack/linuxproc
ReturnToService=1
SlurmctldPort=6817
SlurmdPort=6818
StateSaveLocation=/var/spool/slurm-ctld
SlurmdSpoolDir=/var/spool/slurmd
NodeName=pi[21-38]  CPUs=4  RealMemory=4096  State=UNKNOWN
PartitionName=main Nodes=pi[21-38] Default=YES MaxTime=01:00:00 State=UP
  \end{minted}
  Copiar este archivo a \texttt{/etc/slurm‐llnl/slurm.conf} en todos los nodos.
\end{frame}

%------------------------------ restart ----------------------------
\begin{frame}[fragile]{Reiniciar servicios}
  \begin{minted}{bash}
# En el head
$ sudo systemctl restart slurmctld
# En cada nodo
$ sudo systemctl restart slurmd
  \end{minted}
  \begin{block}{Verificar}
    \begin{minted}{bash}
$ sinfo -Nel
    \end{minted}
  \end{block}
  Los nodos deberían aparecer como \texttt{idle}.
\end{frame}

%------------------------------ actividad -------------------------
\begin{frame}[fragile]{Actividad práctica (30 min)}
  \begin{enumerate}
    \item Instala \texttt{slurmd} en tus dos nodos.
    \item Copia \texttt{munge.key} y \texttt{slurm.conf} desde el head.
    \item Usa \texttt{sinfo} para confirmar que los nodos están en línea.
  \end{enumerate}
\end{frame}

%------------------------------ resumen ---------------------------
\begin{frame}[fragile]{Resumen de la sesión}
  \begin{itemize}
    \item Slurm requiere Munge y un \texttt{slurm.conf} coherente.
    \item Los servicios deben iniciarse tanto en controlador como en nodos.
    \item \texttt{sinfo} es la primera herramienta de diagnóstico.
  \end{itemize}
\end{frame}

\end{document}
