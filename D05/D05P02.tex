% !TEX program = lualatex
\documentclass[aspectratio=169,professionalfonts]{beamer}
%------------------------- Preamble (beamer) -------------------------%
%   Require LuaLaTeX + --shell-escape (for minted)                    %
%--------------------------------------------------------------------%
\usepackage[spanish]{babel}

% ---------- Fonts (modern look) ----------
\usepackage{fontspec}
\defaultfontfeatures{Ligatures=TeX, Scale=MatchLowercase}
\setmainfont{Inter}       % Sans‑serif body (install ttf–Inter)
\setsansfont{Inter}
\setmonofont{Fira Code}   % Nice mono for code

% ---------- Theme ----------
\usetheme[
  titleformat=smallcaps,
  numbering=fraction,
  progressbar=frametitle   % thin progress line under the title
]{metropolis}              % modern beamer theme
\metroset{block=fill}      % coloured blocks

% ---------- Colours ----------
\definecolor{clblue}{HTML}{005F9E}
\definecolor{clorange}{HTML}{FF9130}
\definecolor{clgrey}{HTML}{F4F4F4}

% ---------- minted ----------
\usepackage{minted}
\setminted{
  fontsize=\footnotesize,
  autogobble,
  breaklines,
  tabsize=2,
  style=monokai,
  linenos
}

% ---------- Custom boxes (tcolorbox) ----------
\usepackage{tcolorbox}
\tcbuselibrary{skins, breakable}
% Warning box
\newtcolorbox{warnbox}{
  colback=clorange!10,
  colframe=clorange!80!black,
  breakable,
  title={\faWarning\quad Advertencia},
  fonttitle=\bfseries,
  left=2mm, right=2mm, top=1mm, bottom=1mm
}
% Information box
\newtcolorbox{infobox}{
  colback=clblue!5,
  colframe=clblue!60!black,
  breakable,
  title={\faInfoCircle\quad Nota},
  fonttitle=\bfseries,
  left=2mm, right=2mm, top=1mm, bottom=1mm
}

% ---------- Icons ----------
\usepackage{fontawesome5}

% ---------- Misc tweaks ----------
\graphicspath{{figures/}}
\setlength{\parskip}{0.5em}
\addto\extrasspanish{\renewcommand{\contentsname}{Índice}}
%--------------------------------------------------------------------%



\title[ClústerLab • Día 5]{Primeros trabajos en Slurm}
\subtitle{sbatch, squeue y monitoreo básico}
\author{Equipo docente ClústerLab}
\date{13 de agosto de 2025}

\begin{document}

%------------------------------ portada ----------------------------
\begin{frame}[plain]
  \titlepage
\end{frame}

%---------------------------- AGENDA ----------------------------
\begin{frame}{Agenda}
\begin{itemize}
  \item 01 — ECG filtrado (biomédica)
  \item 02 — Onda 2D tipo FDTD (telecom/ing.)
  \item 03 — Brazo robótico 2D (mecatrónica)
  \item 04 — Monte Carlo para \(\pi\) (general)
  \item 05 — Difusión 2D (física/ing.)
  \item 06 — CNN imágenes (TensorFlow)
  \item 07 — N-body \(O(N^2)\) (física)
  \item Cierre: parámetros y comparación
\end{itemize}
\end{frame}

%================================================================
\section{01 — ECG filtrado}
%================================================================

\begin{frame}{Propósito y contexto}
\begin{itemize}
  \item Generar una señal ECG sintética ruidosa y aplicar \textbf{dos filtros}:
  media móvil y pasa-bajos por FFT.
  \item \textbf{Objetivo de benchmark:} escalar longitud de la señal (\texttt{--size})
  y medir el tiempo por etapa.
  \item \textbf{Aplicación:} preprocesamiento en biomédica (limpieza de señal).
\end{itemize}
\end{frame}

\begin{frame}[fragile]{Fragmento de código clave}
\begin{minted}{python}
def moving_average(x, w=21):
    import numpy as np
    kernel = np.ones(w)/w
    return np.convolve(x, kernel, mode='same')

def lowpass_fft(x, fs, cutoff=20.0):
    import numpy as np
    n = x.size
    freqs = np.fft.rfftfreq(n, d=1.0/fs)
    X = np.fft.rfft(x)
    X[freqs > cutoff] = 0
    return np.fft.irfft(X, n=n)
\end{minted}
\end{frame}

\begin{frame}{Importancia y benchmarks}
\begin{itemize}
  \item Evalúa \textbf{costos de convolución y FFT} al crecer \texttt{--size}.
  \item Métrica: tiempo por etapa (\texttt{generate}, \texttt{moving\_average}, \texttt{fft\_lowpass}).
  \item Resultado esperado: \(\text{FFT} \sim O(n \log n)\), media móvil \(\sim O(n)\).
\end{itemize}
\end{frame}

%================================================================
\section{02 — Onda 2D tipo FDTD}
%================================================================

\begin{frame}{Propósito y contexto}
\begin{itemize}
  \item Evolución simplificada de una onda 2D con esquema de 5 puntos.
  \item \textbf{Objetivo de benchmark:} variar \texttt{--n} (tamaño de malla) y \texttt{--steps}.
  \item \textbf{Aplicación:} prototipos de propagación EM/acústica (telecom/ing.).
\end{itemize}
\end{frame}

\begin{frame}[fragile]{Fragmento de código clave}
\begin{minted}{python}
def step_wave(u, u_prev, c2, dt2_dx2, dt2_dy2):
    up = np.empty_like(u)
    up[1:-1,1:-1] = (2*u[1:-1,1:-1] - u_prev[1:-1,1:-1] +
      c2*((u[2:,1:-1]-2*u[1:-1,1:-1]+u[:-2,1:-1])*dt2_dx2 +
          (u[1:-1,2:]-2*u[1:-1,1:-1]+u[1:-1,:-2])*dt2_dy2))
    up[0,:]=up[-1,:]=up[:,0]=up[:,-1]=0
    return up
\end{minted}
\end{frame}

\begin{frame}{Importancia y benchmarks}
\begin{itemize}
  \item Patrón clásico de stencil: acceso a memoria y aritmética local.
  \item \textbf{Escalado:} cuadrático con \(\texttt{n}^2 \times \texttt{steps}\).
  \item Medir tiempo serial vs. “trabajo paralelo por carga” (tareas independientes).
\end{itemize}
\end{frame}

%================================================================
\section{03 — Brazo Robótico 2D}
%================================================================

\begin{frame}{Propósito y contexto}
\begin{itemize}
  \item Cinemática inversa planar de 2 eslabones para \textbf{muchos} objetivos.
  \item \textbf{Objetivo de benchmark:} variar \texttt{--size} (número de objetivos) y \texttt{--workers}.
  \item \textbf{Aplicación:} trayectoria/planeación en mecatrónica.
\end{itemize}
\end{frame}

\begin{frame}[fragile]{Fragmento de código clave}
\begin{minted}{python}
def ik_2link(target, L1=1.0, L2=1.0):
    import math
    x, y = target
    r2 = x*x + y*y
    c2 = (r2 - L1*L1 - L2*L2) / (2*L1*L2)
    c2 = max(-1.0, min(1.0, c2))
    s2 = math.sqrt(max(0.0, 1.0 - c2*c2))
    th2 = math.atan2(s2, c2)
    k1, k2 = L1 + L2*c2, L2*s2
    th1 = math.atan2(y, x) - math.atan2(k2, k1)
    return th1, th2
\end{minted}
\end{frame}

\begin{frame}{Importancia y benchmarks}
\begin{itemize}
  \item Problema masivo “\(\times\) muchos casos”: perfecto para \textbf{map paralelo}.
  \item Medir aceleración con \texttt{--workers} y \texttt{--size}.
  \item Reportar tiempo/solución y throughput (sol/s).
\end{itemize}
\end{frame}

%================================================================
\section{04 — Monte Carlo para \(\pi\)}
%================================================================

\begin{frame}{Propósito y contexto}
\begin{itemize}
  \item Estimación de \(\pi\) por \textit{Monte Carlo} con muestreo uniforme.
  \item \textbf{Objetivo de benchmark:} variar \texttt{--n} (muestras) y \texttt{--workers}.
  \item \textbf{Aplicación:} patrón de cómputo independiente (embarrassingly parallel).
\end{itemize}
\end{frame}

\begin{frame}[fragile]{Fragmento de código clave}
\begin{minted}{python}
def hits(m):
    import random
    h = 0
    rnd = random.Random()
    for _ in range(m):
        x, y = rnd.random(), rnd.random()
        h += 1 if x*x + y*y <= 1.0 else 0
    return h
\end{minted}
\end{frame}

\begin{frame}{Importancia y benchmarks}
\begin{itemize}
  \item Escala casi lineal con \#procesos (overhead bajo).
  \item Medir precisión vs. costo: error \(\sim 1/\sqrt{n}\).
  \item Excelente demostración de speedup y eficiencia paralela.
\end{itemize}
\end{frame}

%================================================================
\section{05 — Difusión 2D}
%================================================================

\begin{frame}{Propósito y contexto}
\begin{itemize}
  \item Resolver una versión sencilla de la ecuación de difusión en 2D.
  \item \textbf{Objetivo de benchmark:} variar \texttt{--n} y \texttt{--steps}.
  \item \textbf{Aplicación:} transporte de calor/masa, prototipos rápidos.
\end{itemize}
\end{frame}

\begin{frame}[fragile]{Fragmento de código clave}
\begin{minted}{python}
def step_diff(u, alpha=0.2):
    up = np.empty_like(u)
    up[1:-1,1:-1] = (u[1:-1,1:-1] + alpha * (
        u[2:,1:-1] + u[:-2,1:-1] + u[1:-1,2:] + u[1:-1,:-2]
        - 4*u[1:-1,1:-1]))
    up[0,:]=up[-1,:]=up[:,0]=up[:,-1]=0
    return up
\end{minted}
\end{frame}

\begin{frame}{Importancia y benchmarks}
\begin{itemize}
  \item Similar a onda 2D, pero estable y útil para validar pipelines.
  \item Tiempo crece con área de malla \(\times\) pasos.
  \item Benchmark “por carga” para ver escalado en procesos.
\end{itemize}
\end{frame}

%================================================================
\section{06 — CNN con TensorFlow}
%================================================================

\begin{frame}{Propósito y contexto}
\begin{itemize}
  \item Entrenamiento de una \textbf{CNN pequeña} (MNIST o datos sintéticos).
  \item \textbf{Objetivo de benchmark:} variar \texttt{--batch}, \texttt{--epochs} y recurso (CPU/GPU).
  \item \textbf{Aplicación:} visión computacional básica (biomédica/física/ing.).
\end{itemize}
\end{frame}

\begin{frame}[fragile]{Fragmento de código clave}
\begin{minted}{python}
model = tf.keras.Sequential([
  tf.keras.layers.Conv2D(16,3,activation='relu',input_shape=(28,28,1)),
  tf.keras.layers.MaxPooling2D(),
  tf.keras.layers.Conv2D(32,3,activation='relu'),
  tf.keras.layers.MaxPooling2D(),
  tf.keras.layers.Flatten(),
  tf.keras.layers.Dense(64, activation='relu'),
  tf.keras.layers.Dense(10, activation='softmax')
])
model.compile(optimizer='adam',
              loss='sparse_categorical_crossentropy',
              metrics=['accuracy'])
\end{minted}
\end{frame}

\begin{frame}{Importancia y benchmarks}
\begin{itemize}
  \item Demuestra el \textbf{pipeline de datos} y el impacto de \texttt{batch size}.
  \item Comparar CPU vs. GPU si está disponible; medir \emph{throughput} (img/s).
  \item Opción sintética para no depender de descargas en el clúster.
\end{itemize}
\end{frame}

%================================================================
\section{07 — N-body \(O(N^2)\)}
%================================================================

\begin{frame}{Propósito y contexto}
\begin{itemize}
  \item Sistema de partículas con interacción newtoniana simplificada.
  \item \textbf{Objetivo de benchmark:} variar \texttt{--n} (nº partículas) y \texttt{--steps}.
  \item \textbf{Aplicación:} cargas de cómputo intensivas en física/ing.
\end{itemize}
\end{frame}

\begin{frame}[fragile]{Fragmento de código clave}
\begin{minted}{python}
def forces(positions, G=1.0, eps=1e-3):
    n = positions.shape[0]
    acc = np.zeros_like(positions)
    for i in range(n):
        diff = positions - positions[i]
        r2 = (diff**2).sum(axis=1) + eps
        inv = 1.0 / np.sqrt(r2**3)
        inv[i] = 0.0
        acc[i] = (G * inv)[:, None] * diff.sum(axis=0)
    return acc
\end{minted}
\end{frame}

\begin{frame}{Importancia y benchmarks}
\begin{itemize}
  \item Costo \(\sim O(N^2)\): crece muy rápido con \texttt{--n}.
  \item Útil para mostrar cuándo \textbf{sí} conviene paralelizar y cómo trocear trabajo.
  \item Mostrar tiempo por paso y por partícula.
\end{itemize}
\end{frame}

%================================================================
\section{Cierre}
%================================================================

\begin{frame}{Parámetros y comparación resumida}
\small
\begin{itemize}
  \item \textbf{01 ECG:} \texttt{--size} (muestras). Métricas: tiempo por filtro.
  \item \textbf{02 Onda 2D:} \texttt{--n}, \texttt{--steps}. Stencil, memoria.
  \item \textbf{03 IK:} \texttt{--size}, \texttt{--workers}. Throughput (sol/s).
  \item \textbf{04 Monte Carlo:} \texttt{--n}, \texttt{--workers}. Speedup casi lineal.
  \item \textbf{05 Difusión 2D:} \texttt{--n}, \texttt{--steps}. Estabilidad y costo.
  \item \textbf{06 CNN TF:} \texttt{--batch}, \texttt{--epochs}. CPU vs. GPU (\emph{img/s}).
  \item \textbf{07 N-body:} \texttt{--n}, \texttt{--steps}. Costo \(O(N^2)\).
\end{itemize}
\vspace{1em}
\textbf{Recomendación de informe}: correr cada configuración 3 veces, reportar media \(\pm\) desviación, graficar \textit{speedup} y \textit{efficiency}.
\end{frame}

\begin{frame}{Siguientes pasos}
\begin{itemize}
  \item Integrar \texttt{sbatch/srun} en el clúster (plantillas por proyecto).
  \item Automatizar barridos (\texttt{--n}, \texttt{--steps}, \texttt{--workers}, \texttt{--batch}).
  \item Generar tablas/figuras comparativas y \textbf{presentación final por equipo}.
\end{itemize}
\end{frame}

\end{document}
