% !TEX program = lualatex
\documentclass[aspectratio=169,professionalfonts]{beamer}
%------------------------- Preamble (beamer) -------------------------%
%   Require LuaLaTeX + --shell-escape (for minted)                    %
%--------------------------------------------------------------------%
\usepackage[spanish]{babel}

% ---------- Fonts (modern look) ----------
\usepackage{fontspec}
\defaultfontfeatures{Ligatures=TeX, Scale=MatchLowercase}
\setmainfont{Inter}       % Sans‑serif body (install ttf–Inter)
\setsansfont{Inter}
\setmonofont{Fira Code}   % Nice mono for code

% ---------- Theme ----------
\usetheme[
  titleformat=smallcaps,
  numbering=fraction,
  progressbar=frametitle   % thin progress line under the title
]{metropolis}              % modern beamer theme
\metroset{block=fill}      % coloured blocks

% ---------- Colours ----------
\definecolor{clblue}{HTML}{005F9E}
\definecolor{clorange}{HTML}{FF9130}
\definecolor{clgrey}{HTML}{F4F4F4}

% ---------- minted ----------
\usepackage{minted}
\setminted{
  fontsize=\footnotesize,
  autogobble,
  breaklines,
  tabsize=2,
  style=monokai,
  linenos
}

% ---------- Custom boxes (tcolorbox) ----------
\usepackage{tcolorbox}
\tcbuselibrary{skins, breakable}
% Warning box
\newtcolorbox{warnbox}{
  colback=clorange!10,
  colframe=clorange!80!black,
  breakable,
  title={\faWarning\quad Advertencia},
  fonttitle=\bfseries,
  left=2mm, right=2mm, top=1mm, bottom=1mm
}
% Information box
\newtcolorbox{infobox}{
  colback=clblue!5,
  colframe=clblue!60!black,
  breakable,
  title={\faInfoCircle\quad Nota},
  fonttitle=\bfseries,
  left=2mm, right=2mm, top=1mm, bottom=1mm
}

% ---------- Icons ----------
\usepackage{fontawesome5}

% ---------- Misc tweaks ----------
\graphicspath{{figures/}}
\setlength{\parskip}{0.5em}
\addto\extrasspanish{\renewcommand{\contentsname}{Índice}}
%--------------------------------------------------------------------%



\title[ClústerLab • Día 5]{Primeros trabajos en Slurm}
\subtitle{sbatch, squeue y monitoreo básico}
\author{Equipo docente ClústerLab}
\date{13 de agosto de 2025}

\begin{document}

%------------------------------ portada ----------------------------
\begin{frame}[plain]
  \titlepage
\end{frame}

%------------------------------ objetivos --------------------------
\begin{frame}[fragile]{Objetivos de la sesión}
  \begin{itemize}
    \item Enviar el primer trabajo en lote con \texttt{sbatch}.
    \item Consultar la cola con \texttt{squeue} y \texttt{sacct}.
    \item Interpretar estados: \texttt{PENDING}, \texttt{RUNNING}, \texttt{COMPLETED}.
    \item Introducir arreglos de trabajos y límites de recursos.
  \end{itemize}
\end{frame}

%------------------------------ agenda -----------------------------
\begin{frame}[fragile]{Agenda (11:30 – 13:00)}
  \begin{enumerate}
    \item Script \texttt{sbatch} mínimo.
    \item Opciones comunes: \texttt{-p}, \texttt{-N}, \texttt{-n}, \texttt{--mem}, \texttt{--time}.
    \item Seguimiento con \texttt{squeue} y \texttt{sacct}.
    \item Ejemplo Python: calcular \(\pi\) con Monte‑Carlo.
    \item Actividad práctica grupal.
  \end{enumerate}
\end{frame}

%------------------------------ sbatch script ----------------------
\begin{frame}[fragile]{Script \texttt{hello.slurm}}
  \begin{minted}[bgcolor=gray!15]{bash}
#!/usr/bin/env bash
#SBATCH -p main      # partición
#SBATCH -N 1         # nodos
#SBATCH -n 4         # tareas
#SBATCH --time=01:00 # límite
#SBATCH --job-name=hello

hostname
sleep 5
  \end{minted}
  Enviar con:
  \begin{minted}{bash}
$ sbatch hello.slurm
  \end{minted}
\end{frame}

%------------------------------ squeue -----------------------------
\begin{frame}[fragile]{Monitorear la cola}
  \begin{minted}{bash}
$ squeue -u $USER
JOBID PARTITION NAME  USER  ST TIME  NODES NODELIST(REASON)
123   main      hello jper  PD 0:00  1     (Resources)
  \end{minted}
  \begin{itemize}
    \item Estado \texttt{PD}: pendiente. \texttt{R}: ejecutándose.
  \end{itemize}
\end{frame}

%------------------------------ sacct ------------------------------
\begin{frame}[fragile]{Resumen de uso con \texttt{sacct}}
  \begin{minted}{bash}
$ sacct -j 123 --format=JobID,State,Elapsed,MaxRSS
  \end{minted}
  Útil para facturación y análisis de memoria.
\end{frame}

%------------------------------ array ------------------------------
\begin{frame}[fragile]{Arreglos de trabajos}
  \begin{minted}{bash}
$ sbatch --array=1-10%3 pi_mc.slurm
  \end{minted}
  Ejecuta 10 réplicas Monte‑Carlo, max 3 simultáneas.
\end{frame}

%------------------------------ python pi --------------------------
\begin{frame}[fragile]{Ejemplo Python: \(\pi\) Monte‑Carlo}
  \begin{minted}[bgcolor=gray!15]{python}
import random, sys
N = int(sys.argv[1])
inside = sum((random.random()**2 + random.random()**2) < 1 for _ in range(N))
print(4 * inside / N)
  \end{minted}
  Script \texttt{pi.slurm}:
  \begin{minted}[bgcolor=gray!15]{bash}
#!/usr/bin/env bash
#SBATCH -p main -n 1 -t 00:02:00 -o pi_%A_%a.out
python3 pi.py 1000000
  \end{minted}
\end{frame}

%------------------------------ actividad -------------------------
\begin{frame}[fragile]{Actividad práctica (30 min)}
  \begin{enumerate}
    \item Envía \texttt{hello.slurm} y confirma ejecución.
    \item Modifica límites de tiempo y observa el cambio en \texttt{squeue}.
    \item Lanza un arreglo Monte‑Carlo de \(\pi\) (10 tareas).
    \item Comparte la desviación estándar de los resultados en el canal.
  \end{enumerate}
\end{frame}

%------------------------------ resumen ---------------------------
\begin{frame}[fragile]{Resumen de la sesión}
  \begin{itemize}
    \item \texttt{sbatch} es la puerta de entrada a Slurm.
    \item \texttt{squeue} y \texttt{sacct} permiten diagnosticar trabajos.
    \item Arreglos facilitan barrer parámetros.
  \end{itemize}
\end{frame}

%------------------------------ recursos --------------------------
\begin{frame}[fragile]{Recursos recomendados}
  \begin{itemize}
    \item \textit{Slurm Quick Start Guide} – SchedMD.
    \item \url{https://slurm.schedmd.com/sbatch.html}.
    \item \url{https://slurm.schedmd.com/squeue.html}.
  \end{itemize}
\end{frame}

\end{document}
