% !TEX program = lualatex
\documentclass[aspectratio=169,professionalfonts]{beamer}
%------------------------- Preamble (beamer) -------------------------%
%   Require LuaLaTeX + --shell-escape (for minted)                    %
%--------------------------------------------------------------------%
\usepackage[spanish]{babel}

% ---------- Fonts (modern look) ----------
\usepackage{fontspec}
\defaultfontfeatures{Ligatures=TeX, Scale=MatchLowercase}
\setmainfont{Inter}       % Sans‑serif body (install ttf–Inter)
\setsansfont{Inter}
\setmonofont{Fira Code}   % Nice mono for code

% ---------- Theme ----------
\usetheme[
  titleformat=smallcaps,
  numbering=fraction,
  progressbar=frametitle   % thin progress line under the title
]{metropolis}              % modern beamer theme
\metroset{block=fill}      % coloured blocks

% ---------- Colours ----------
\definecolor{clblue}{HTML}{005F9E}
\definecolor{clorange}{HTML}{FF9130}
\definecolor{clgrey}{HTML}{F4F4F4}

% ---------- minted ----------
\usepackage{minted}
\setminted{
  fontsize=\footnotesize,
  autogobble,
  breaklines,
  tabsize=2,
  style=monokai,
  linenos
}

% ---------- Custom boxes (tcolorbox) ----------
\usepackage{tcolorbox}
\tcbuselibrary{skins, breakable}
% Warning box
\newtcolorbox{warnbox}{
  colback=clorange!10,
  colframe=clorange!80!black,
  breakable,
  title={\faWarning\quad Advertencia},
  fonttitle=\bfseries,
  left=2mm, right=2mm, top=1mm, bottom=1mm
}
% Information box
\newtcolorbox{infobox}{
  colback=clblue!5,
  colframe=clblue!60!black,
  breakable,
  title={\faInfoCircle\quad Nota},
  fonttitle=\bfseries,
  left=2mm, right=2mm, top=1mm, bottom=1mm
}

% ---------- Icons ----------
\usepackage{fontawesome5}

% ---------- Misc tweaks ----------
\graphicspath{{figures/}}
\setlength{\parskip}{0.5em}
\addto\extrasspanish{\renewcommand{\contentsname}{Índice}}
%--------------------------------------------------------------------%



\title[ClústerLab • Día 6]{MPI con \texttt{mpi4py} y arranque de proyecto}
\subtitle{Hello World, π distribuido y elección de proyecto final}
\author{Equipo docente ClústerLab}
\date{15 de agosto de 2025}

\begin{document}

%------------------------------ portada ----------------------------
\begin{frame}[plain]
  \titlepage
\end{frame}

%------------------------------ objetivos --------------------------
\begin{frame}[fragile]{Objetivos de la sesión}
  \begin{itemize}
    \item Ejecutar programas MPI sencillos con \texttt{mpi4py}.
    \item Comprender comunicación punto‑a‑punto y colectivas básicas.
    \item Distribuir un cálculo Monte‑Carlo de \(\pi\) entre procesos.
    \item Formar equipos y seleccionar tema de proyecto final.
  \end{itemize}
\end{frame}

%------------------------------ agenda -----------------------------
\begin{frame}[fragile]{Agenda (11:30 – 13:00)}
  \begin{enumerate}
    \item Instalación rápida de \texttt{mpi4py}.
    \item \textit{Hello World} en 4 procesos.
    \item Comunicación punto-a-punto: \texttt{send}/\texttt{recv}.
    \item Reducción para estimar \(\pi\).
    \item Discusión de ideas y formación de equipos.
  \end{enumerate}
\end{frame}

%------------------------------ install mpi4py ---------------------
\begin{frame}[fragile]{Instalar \texttt{mpi4py}}
  \begin{minted}{bash}
$ pip install --user mpi4py
  \end{minted}
  Usaremos \texttt{srun} para lanzar varios procesos:
  \begin{minted}{bash}
$ srun -n 4 python3 hello.py
  \end{minted}
\end{frame}

%------------------------------ hello world ------------------------
\begin{frame}[fragile]{\textit{Hello World}}
  \begin{minted}[bgcolor=gray!15]{python}
from mpi4py import MPI
rank = MPI.COMM_WORLD.Get_rank()
size = MPI.COMM_WORLD.Get_size()
print(f"Hola desde rank {rank} de {size}")
  \end{minted}
\end{frame}

%------------------------------ send recv -------------------------
\begin{frame}[fragile]{Comunicación punto‑a‑punto}
  \begin{minted}[bgcolor=gray!15]{python}
from mpi4py import MPI
comm = MPI.COMM_WORLD
rank = comm.Get_rank()
if rank == 0:
    data = {'x': 42}
    comm.send(data, dest=1)
else:
    recv = comm.recv(source=0)
    print('Rank', rank, 'recibió', recv)
  \end{minted}
\end{frame}

%------------------------------ pi distributed --------------------
\begin{frame}[fragile]{Monte‑Carlo de \(\pi\) distribuido}
  \begin{minted}[bgcolor=gray!15]{python}
import random, time
from mpi4py import MPI
comm, rank, size = MPI.COMM_WORLD, *map(MPI.Comm.Get_rank,
                                         [MPI.COMM_WORLD]*1),

N = 1_000_000
local = sum((random.random()**2 + random.random()**2) < 1 for _ in range(N))
inside = comm.reduce(local, op=MPI.SUM, root=0)
if rank == 0:
    pi = 4 * inside / (N * size)
    print('Pi ≈', pi)
  \end{minted}
\end{frame}

%------------------------------ equipos ---------------------------
\begin{frame}[fragile]{Formación de equipos}
  \begin{itemize}
    \item 6 equipos de 3 personas.
    \item Cada equipo selecciona un tema y entrega mini‑propuesta (3 ppts + README).
    \item Presentaciones finales el \textbf{22 de agosto}.
  \end{itemize}
\end{frame}

%------------------------------ ideas proyecto --------------------
\begin{frame}[fragile]{Ideas de proyecto final}
  \begin{enumerate}
    \item \textbf{Fractal Mandelbrot} paralelizado con MPI + rendering video.
    \item \textbf{Conway GPU vs MPI}: comparar velocidades automatizadas.
    \item \textbf{Weather SIM}: stencil 2D de ecuación de difusión de calor.
    \item \textbf{N‑Body} simplificado (gravitación) con OpenMP vs MPI.
    \item \textbf{Regex Search} distribuido en logs grandes con MPI I/O.
    \item \textbf{Monte Carlo Ising 2D}: estimar \(T_c\) con CNN + Slurm arrays.
    \item \textbf{Speedup de BLAS}: medir \texttt{OpenBLAS} vs \texttt{Eigen} en la Pi‑farm.
  \end{enumerate}
\end{frame}

%------------------------------ actividad -------------------------
\begin{frame}[fragile]{Actividad práctica (30 min)}
  \begin{enumerate}
    \item Ejecuta \textit{Hello World} con 4 procesos.
    \item Corre el cálculo de \(\pi\) y compara tiempos vs CPU serial.
    \item Con tu equipo, elige un tema y sube un boceto de objetivos al repositorio.
  \end{enumerate}
\end{frame}

%------------------------------ resumen ---------------------------
\begin{frame}[fragile]{Resumen de la sesión}
  \begin{itemize}
    \item \texttt{mpi4py} ofrece una puerta de entrada sencilla a MPI.
    \item Reducciones y arreglos de trabajos se combinan para experimentos masivos.
    \item Los proyectos finales consolidarán lo aprendido durante la escuela.
  \end{itemize}
\end{frame}

\end{document}
