% !TEX program = lualatex
\documentclass[aspectratio=169,professionalfonts]{beamer}
%------------------------- Preamble (beamer) -------------------------%
%   Require LuaLaTeX + --shell-escape (for minted)                    %
%--------------------------------------------------------------------%
\usepackage[spanish]{babel}

% ---------- Fonts (modern look) ----------
\usepackage{fontspec}
\defaultfontfeatures{Ligatures=TeX, Scale=MatchLowercase}
\setmainfont{Inter}       % Sans‑serif body (install ttf–Inter)
\setsansfont{Inter}
\setmonofont{Fira Code}   % Nice mono for code

% ---------- Theme ----------
\usetheme[
  titleformat=smallcaps,
  numbering=fraction,
  progressbar=frametitle   % thin progress line under the title
]{metropolis}              % modern beamer theme
\metroset{block=fill}      % coloured blocks

% ---------- Colours ----------
\definecolor{clblue}{HTML}{005F9E}
\definecolor{clorange}{HTML}{FF9130}
\definecolor{clgrey}{HTML}{F4F4F4}

% ---------- minted ----------
\usepackage{minted}
\setminted{
  fontsize=\footnotesize,
  autogobble,
  breaklines,
  tabsize=2,
  style=monokai,
  linenos
}

% ---------- Custom boxes (tcolorbox) ----------
\usepackage{tcolorbox}
\tcbuselibrary{skins, breakable}
% Warning box
\newtcolorbox{warnbox}{
  colback=clorange!10,
  colframe=clorange!80!black,
  breakable,
  title={\faWarning\quad Advertencia},
  fonttitle=\bfseries,
  left=2mm, right=2mm, top=1mm, bottom=1mm
}
% Information box
\newtcolorbox{infobox}{
  colback=clblue!5,
  colframe=clblue!60!black,
  breakable,
  title={\faInfoCircle\quad Nota},
  fonttitle=\bfseries,
  left=2mm, right=2mm, top=1mm, bottom=1mm
}

% ---------- Icons ----------
\usepackage{fontawesome5}

% ---------- Misc tweaks ----------
\graphicspath{{figures/}}
\setlength{\parskip}{0.5em}
\addto\extrasspanish{\renewcommand{\contentsname}{Índice}}
%--------------------------------------------------------------------%



\title[ClústerLab • Día 6]{Automatización y benchmarking}
\subtitle{systemctl · job arrays · desempeño de matrices}
\author{Equipo docente ClústerLab}
\date{15 de agosto de 2025}

\begin{document}

%----------------------------- portada ----------------------------
\begin{frame}[plain]
  \titlepage
\end{frame}

%----------------------------- objetivos --------------------------
\begin{frame}[fragile]{Objetivos de la sesión}
  \begin{itemize}
    \item Controlar servicios con \texttt{systemctl} (Slurm, NFS, Munge).
    \item Construir job arrays paramétricos para pruebas automáticas.
    \item Medir el rendimiento de multiplicación de matrices y graficar resultados.
  \end{itemize}
\end{frame}

%----------------------------- agenda -----------------------------
\begin{frame}[fragile]{Agenda (09:00 – 11:00)}
  \begin{enumerate}
    \item Recordatorio de servicios críticos.
    \item \texttt{systemctl}: enable, start, status.
    \item Job array de ejemplo con tamaños de matriz.
    \item Script Python benchmarking + Matplotlib.
    \item Actividad práctica.
  \end{enumerate}
\end{frame}

%----------------------------- systemctl --------------------------
\begin{frame}[fragile]{Gestión de servicios con \texttt{systemctl}}
  \begin{minted}{bash}
# Habilitar Slurm al arranque
$ sudo systemctl enable slurmd
$ sudo systemctl enable slurmctld

# Comprobar estado
$ systemctl status slurmd --no-pager
  \end{minted}
  \pause
  Otros servicios importantes:
  \begin{itemize}
    \item \textbf{munge} – autenticación.
    \item \textbf{nfs-server} – almacenamiento compartido.
    \item \textbf{chrony} – sincronización de reloj.
  \end{itemize}
\end{frame}

%----------------------------- job array desc ---------------------
\begin{frame}[fragile]{Job array paramétrico}
  \begin{itemize}
    \item Queremos barrer tamaños de matriz \(N = 256,512,1024,2048\).
    \item Cada tarea recibe un índice y lo traduce a \(N\).
  \end{itemize}
  \begin{minted}{bash}
#SBATCH --array=0-3
SIZES=(256 512 1024 2048)
N=${SIZES[$SLURM_ARRAY_TASK_ID]}
python3 matmul_bench.py $N
  \end{minted}
\end{frame}

%----------------------------- matmul python ----------------------
\begin{frame}[fragile]{Script \texttt{matmul\_bench.py}}
  \begin{minted}[bgcolor=gray!15]{python}
import sys, time, numpy as np
N = int(sys.argv[1])
A = np.random.rand(N, N).astype(np.float32)
B = np.random.rand(N, N).astype(np.float32)
start = time.perf_counter()
C = A @ B  # multiplicación
elapsed = time.perf_counter() - start
print(N, elapsed)
  \end{minted}
  Salida se redirige a \texttt{results.txt} para cada tarea.
\end{frame}

%----------------------------- plot script ------------------------
\begin{frame}[fragile]{Graficar resultados}
  \begin{minted}[bgcolor=gray!15]{python}
import pandas as pd, matplotlib.pyplot as plt

df = pd.read_csv('results.txt', delim_whitespace=True,
                 names=['N','time'])
plt.plot(df['N'], df['time'], marker='o')
plt.xlabel('Tamaño de matriz N')
plt.ylabel('Tiempo (s)')
plt.title('Benchmark matmul')
plt.xscale('log'); plt.yscale('log')
plt.savefig('matmul.png')
  \end{minted}
  Se puede ejecutar en el nodo head tras completarse el arreglo.
\end{frame}

%----------------------------- actividad -------------------------
\begin{frame}[fragile]{Actividad práctica (30 min)}
  \begin{enumerate}
    \item Activa y revisa los servicios \texttt{slurmd} y \texttt{nfs-server} con \texttt{systemctl}.
    \item Envía el arreglo de multiplicación de matrices.
    \item Genera el gráfico \texttt{matmul.png} y sube la imagen.
  \end{enumerate}
\end{frame}

%----------------------------- resumen ---------------------------
\begin{frame}[fragile]{Resumen de la sesión}
  \begin{itemize}
    \item \texttt{systemctl} facilita la administración persistente de servicios.
    \item Job arrays = automatización + reproducibilidad.
    \item Benchmarks cuantifican la ganancia de paralelismo y optimizaciones.
  \end{itemize}
\end{frame}

\end{document}
