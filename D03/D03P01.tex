% !TEX program = lualatex
\documentclass[aspectratio=169, professionalfonts]{beamer}
\usepackage[spanish]{babel}
\usepackage{fontspec}
\usepackage{graphicx}
\usepackage{mwe}
\usepackage{xcolor}
\usepackage{emoji}
\usepackage{minted}
\setminted{
  fontsize=\small,
  bgcolor=gray!10,
  linenos=false
}
\setmainfont{Libertinus Serif}

\title[ClústerLab • Día 3]{ClústerLab 2025\\\huge Raspberry Pi OS y monitoreo básico}
\subtitle{Escuela de Computación Distribuida – ITM Medellín}
\author{Equipo docente ClústerLab}
\date{8 de agosto 2025}

\begin{document}

%----------------------------------------------------------------- title
\begin{frame}[plain]
  \titlepage
\end{frame}

%----------------------------------------------------------------- agenda
\begin{frame}{Agenda Día 3 (09:00 – 13:00)}
\begin{enumerate}
  \item 09:00–09:15   Revisión rápida (IP + SSH).  
  \item 09:15–10:00   Raspberry Pi OS Lite vs Desktop.  
  \item 10:00–10:45   Cambiar hostname, crear usuarios, sudo.  
  \item 10:45–11:30   Actualizar sistema y activar servicios.  
  \item 11:30–12:30   Herramientas de monitoreo: \texttt{htop}, \texttt{df}, temperatura CPU.  
  \item 12:30–13:00   Reto grupal: nodo accesible por nombre + script diagnóstico.  
\end{enumerate}
\end{frame}

%----------------------------------------------------------------- lite vs desktop
\begin{frame}{Raspberry Pi OS Lite vs Desktop}
\begin{columns}[T]
\begin{column}{0.55\textwidth}
\begin{itemize}
  \item \textbf{Lite}: sin entorno gráfico, arranque más rápido, menos RAM ≈ 300 MB.  
  \item \textbf{Desktop}: incluye GUI, útil para demostraciones locales.  
  \item Para el clúster usaremos **Lite** → menos sobre-carga.  
\end{itemize}
\end{column}
\begin{column}{0.4\textwidth}
  \centering\includegraphics[width=\linewidth]{example-image}
\end{column}
\end{columns}
\end{frame}

%----------------------------------------------------------------- hostname
\begin{frame}[fragile]{Cambiar hostname}
\begin{minted}{bash}
# Editar hostname
$ sudo nano /etc/hostname
nodo01        # <--- nuevo nombre

# Editar hosts para resolver localmente
$ sudo nano /etc/hosts
127.0.1.1   nodo01
\end{minted}
Tras reiniciar (\texttt{sudo reboot}) el prompt mostrará el nuevo nombre.
\end{frame}

%----------------------------------------------------------------- usuarios
\begin{frame}[fragile]{Crear usuario y dar sudo}
\begin{minted}{bash}
# Añadir usuario alumno
$ sudo adduser alumno
# Agregar a grupo sudo
$ sudo usermod -aG sudo alumno
\end{minted}
\begin{itemize}
  \item Buen hábito: desactivar login del usuario \texttt{pi} cuando todo funcione.  
  \item Ver grupos del usuario: \texttt{groups alumno}.  
\end{itemize}
\end{frame}

%----------------------------------------------------------------- update & raspi-config
\begin{frame}[fragile]{Actualizar sistema y raspi-config}
\begin{minted}{bash}
$ sudo apt update && sudo apt full-upgrade -y
$ sudo raspi-config      # menú de configuración
\end{minted}
\begin{itemize}
  \item Cambiar zona horaria y localización.  
  \item Habilitar \texttt{ssh} si se desactivó.  
  \item Expandir sistema de archivos (normalmente ya expandido).  
\end{itemize}
\end{frame}

%----------------------------------------------------------------- htop / df
\begin{frame}[fragile]{Monitoreo rápido}
\begin{columns}[T]
\begin{column}{0.52\textwidth}
\texttt{htop} (CPU, RAM, procesos)  
\begin{minted}{bash}
$ sudo apt install htop
$ htop
\end{minted}

\texttt{df -h} (uso de disco)  
\begin{minted}{bash}
$ df -h
\end{minted}
\end{column}
\begin{column}{0.43\textwidth}
  \includegraphics[width=\linewidth]{example-image-b}
\end{column}
\end{columns}
\end{frame}

%----------------------------------------------------------------- temperatura
\begin{frame}[fragile]{Temperatura del CPU y voltaje}
\begin{minted}{bash}
$ vcgencmd measure_temp
temp=42.0'C
$ vcgencmd measure_volts
volt=0.86V
\end{minted}
Para monitorizar continuamente:  
\begin{minted}{bash}
watch -n 2 vcgencmd measure_temp
\end{minted}
\end{frame}

%----------------------------------------------------------------- acceso remoto
\begin{frame}{Acceso remoto por nombre de host}
\begin{itemize}
  \item Linux/macOS → mDNS (\textit{hostname.local}) funciona por defecto.  
  \item En Windows instala \emph{Bonjour Print Services} o usa un archivo \texttt{C:\textbackslash Windows\textbackslash System32\textbackslash drivers\textbackslash etc\textbackslash hosts}.  
  \item Objetivo: \texttt{ssh alumno@nodo01.local} sin conocer la IP.  
\end{itemize}
\end{frame}

%----------------------------------------------------------------- reto grupal
\begin{frame}[fragile]{Reto grupal (12:30 – 13:00)}
\begin{enumerate}
  \item Cambiar hostname → \texttt{nodoXX}.  
  \item Configurar acceso SSH por clave, sin contraseña.  
  \item Escribir \texttt{diag.sh} que muestre:
    \begin{itemize}
      \item Hostname + temperatura + uso CPU.  
      \item Espacio libre en \texttt{/home}.  
    \end{itemize}
  \item Compartir el script con otra Pi y ejecutarlo remotamente.  
\end{enumerate}
\end{frame}

%----------------------------------------------------------------- cierre
\begin{frame}[fragile]
Nodo listo y monitoreado ✔\\
\small ¡Fin de la Semana 1!
\end{frame}

\end{document}

